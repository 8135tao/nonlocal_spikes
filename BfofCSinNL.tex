\documentclass[letterpaper,11pt]{article}

\usepackage{ucs}
\usepackage[utf8x]{inputenc}
\usepackage{graphicx}
\usepackage{amsfonts}
\usepackage{dsfont}
\usepackage{amssymb}
\usepackage{amsmath,mathrsfs}
\usepackage{amsthm}
\usepackage{enumerate}
\usepackage{stmaryrd}
\usepackage{fullpage}
\usepackage{ifthen}
\usepackage{subfigure}
\usepackage{epic}
\usepackage{authblk}
\usepackage{textcomp}
\usepackage[small]{caption}
%\usepackage{mathtools}


\usepackage[hypertexnames=false,colorlinks=true,linkcolor=blue,citecolor=blue]{hyperref}
\usepackage[numbers,comma,square,sort&compress]{natbib}
\usepackage[letterpaper,text={7in,9in},centering]{geometry}


\usepackage{color}
\usepackage{titlesec}
\setlength{\parindent}{0.0in}
\setlength{\parskip}{1.0ex plus0.2ex minus0.2ex}
\renewcommand{\baselinestretch}{1.1}
\graphicspath{{eps/}{pdf/}}
%\setcaptionmargin{0.25in}
\def\captionfont{\itshape\small}
\def\captionlabelfont{\upshape\small}

\renewcommand{\labelenumi}{(\roman{enumi})}

\newcommand{\bqq}{\begin{equation}}
\newcommand{\eqq}{\end{equation}}
\newcommand{\bqs}{\begin{equation*}}
\newcommand{\eqs}{\end{equation*}}

\newcommand{\C}{\mathbb{C}}
\newcommand{\D}{\mathbb{D}}
\newcommand{\N}{\mathbb{N}}
\newcommand{\R}{\mathbb{R}} 
\newcommand{\Z}{\mathbb{Z}}

\newcommand{\rme}{\mathrm{e}}
\newcommand{\rmi}{\mathrm{i}}
\newcommand{\rmd}{\mathrm{d}}
\newcommand{\rmo}{{\scriptstyle\mathcal{O}}}
\newcommand{\rmO}{\mathcal{O}}
\newcommand{\eps}{\varepsilon}
\newcommand{\B}{\mathcal{B}}
\newcommand{\Rm}{\mathcal{R}}
\newcommand{\Nl}{\mathcal{N}}
\newcommand{\K}{\mathcal{K}}
\newcommand{\G}{\mathcal{G}}
\newcommand{\F}{\mathcal{F}}
\newcommand{\M}{\mathcal{M}}
\newcommand{\cS}{\mathcal{S}}
\newcommand{\cL}{\mathcal{L}}

\newcommand{\diag}{\operatorname{diag}}


\numberwithin{equation}{section}

\newenvironment{Hypothesis}[1]%
  {\begin{trivlist}\item[]{\bf Hypothesis #1 }\em}{\end{trivlist}}

\renewcommand{\arraystretch}{1.25}


% Define Theorem Styles ----------------------------------
\theoremstyle{plain}
\newtheorem{theorem}{Theorem}[section]
\newtheorem{proposition}[theorem]{Proposition}
\newtheorem{lemma}[theorem]{Lemma}
\newtheorem{corollary}[theorem]{Corollary}
\newtheorem{conjecture}[theorem]{Conjecture}
\newtheorem{main}[theorem]{Main Result}
\newtheorem{rmk}[theorem]{rmk}
\theoremstyle{remark}
\newtheorem*{remark}{Remark}

\newcommand{\etal}{\textit{et al.}\ }

\newcommand{\greg}[1]{%
  {\color{blue}\textbf{Greg:} #1}%
 }
 
\newcommand{\arnd}[1]{%
  {\color{red}\textbf{Arnd:} #1}%
 }

\newenvironment{Proof}[1][.]%
 {\begin{trivlist}\item[]\textbf{Proof#1 }}%
 {\hspace*{\fill}$\rule{0.3\baselineskip}{0.35\baselineskip}$\end{trivlist}}

\renewcommand\labelitemi{$\bullet$}


\title{Bifurcation of coherent structures in nonlocally coupled system}
\author{Arnd Scheel and Tianyu Tao}
\date{2017}
\begin{document}
\maketitle
\begin{abstract}

Motivated by models for neural fields, we study the existence of pulses  bifurcating from a spatially homogeneous state in nonlocally coupled systems of equations. More specifically, we look at equations of the form $U + \K\ast U = \Nl(U;\mu)$, where $\Nl$ encodes nonlinear terms, $\K$ is an even matrix convolution kernel. Assuming the presence of neutral modes, that is, solutions of the form $u\sim \exp(i \ell x)$ to the linear part, we show under appropriate assumptions on the nonlinearity and the unfolding in $\mu$ that pulses bifurcate. Such an analysis is carried out using center manifold reduction, when coupling is local, say, $\K=\delta''$. Here, we rely on functional analytic methods using predictors from formal expansions and correctors obtained after preconditioning the nonlinear system.
\end{abstract}

\section{Introduction}
In this paper we study the equation
\begin{equation} \label{system}
U+\K\ast U = \Nl(U;\mu) ,
\end{equation}
where $U=U(x):\R \to \R^n$, and $\K\ast U$ stands for matrix convolution,
\[
(\K\ast U(x))_i = \sum_{j=1}^m \int_\R \K_{i,j}(x-y)U_j(y)dy, \hspace{0.2in} 1\le i\le m,
\]

and $\Nl(U;\mu)$ encodes nonlinear terms which depend on a parameter $\mu\ge 0$. We assume $\Nl(0;\mu) =0$, so $U\equiv 0$ is a trivial solution for all $\mu$.

We are interested in possible solutions that bifurcate from the trivial solution $U\equiv 0$ as one increases the parameter $\mu$. The existence of such solutions bears important physical meaning and poses interesting mathematical questions.

One of the examples arises when studying stationary or traveling-wave solutions to neural field equations which are typically modeled by equations of the form
\begin{equation}\label{NFE}
\frac{du}{dt} =- u+\K\ast S(u),
\end{equation}
where $u=u(t,x) \in \R$ represents the local activity of a population of neurons at position $x$ in the cortex, and $S$ is a firing rate function, the kernel $\K$ encodes the connectivity. Stationary solution to \eqref{NFE} are thought to be associated to short term memory, providing motivation for extensive studies of such solutions.

Another area where nonlocal equations such as \eqref{system} appear is when studying phase transition. Examples include equation of the form
\begin{equation}\label{PT}
u_t = -u+J\ast u - f(u), 
\end{equation}
where now $u = u(x,t) \in \R$ is an order parameter describing the state of a solid material at position $x$ and time $t$, $J$ is a convolution kernel with $\int_\R J=1$ and $f$ is a bistable nonlinearity with zeros $-1,1$ and $\alpha \in (-1,1)$. As a consequence, $u = \pm 1$ are steady state solution to \eqref{PT}, they may represent two orientations of a perfect crystal. Any values between $u=\pm 1$ can represent intermediate states of the crystal. For example, a traveling wave solution $u(x)$ connecting $u=-1$ and $u=+1$ represents the process of ``invasion'' of one pure state into the other. The advantage of using the nonlocal dispersal term $-u+J\ast u$ instead of the diffusion term $u_{xx}$ is that more general types of interactions between states at nearby locations in the medium can be accounted for.

In terms of techniques, one popular approach is to use a special form of the kernel $\K$ that has ``rational'' Fourier transforms: and reduces the nonlocal equation to a system of ordinary differential equation (ODE), where classical dynamical system methods are available.

As a specific example, spatially localized solutions (referred to as ``bumps'' or ``homoclinics'') in equations of the form 
\begin{equation}\label{Laing}
\partial_t a(x,t) = -a(x,t)+\int_{-\infty}^{\infty} w(x-y)S(\mu a(y,t))dy,
\end{equation}
were investigated. Laing et al. \cite{laing2003pde} considered the case $S(\cdot) = 2\exp(-r/(\cdot-\theta)^2)H(\cdot-\theta)$, with $H$ the Heaviside function and $w(x) = \exp(-b|x|)(b\sin |x|+\cos x)$. 
Faye et al. \citep{faye2013localized} studied the case $S(\cdot) = (1+e^{-\cdot+\theta})^{-1}$ and general $w$ such that $\widehat{w}(\xi) = R(\xi^2)/Q(\xi^2)$, where $R,Q$ are polynomials in $\xi^2$ satisfying $\deg R<\deg Q$. Using this approach, they are able to transform the nerual field intego-differential equation into a partial differential equation (PDE). Restricting to stationary solutions, the PDE reduces further into a system of ODE. Laing et al. \citep{laing2003pde} proceed with numerical investigations, whereas Faye et al. \citep{faye2013localized} used normal form theory and center manifold reduction (see for example \cite{haragus2010local}) to find the desired solution.

Another example is in \cite{faye2013existence}, where Faye considered a variant of the FitzHugh–Nagumo equation of the form
\begin{subequations}
\begin{eqnarray}
\tau u_t(x,t) &=& -u(x,t)+\int_\R J(x-y)q(y,t)S(u(y,t))dy, \\
\eps^{-1} q_t(x,t) &=& 1-q(x,t)-\beta q(x,t) S(u(x,t)) ,
\end{eqnarray}
\end{subequations}
with $S(u) = (1+e^{-\lambda(u-\kappa})^{-1}, J(x) = b/2\exp(-b|x|)$, and parameters $\lambda,\kappa,b$. Using the property $\widehat{J}(\ell) = b^2/(b^2+\ell^2)$ and looking for traveling wave solutions, Faye was able to reduce the neural field equation into a system of ODEs with a singular perturbation structure. He then used geometric singular perturbation theory (GSPT) to find the desired solution.

Other important techniques for demonstrating the existence of travelling waves, not necessarily spatially localized (refered to as ``fronts'' and ``backs''), include homotopy arguments. In \cite{Bates1997}, Bates et al. established a homotopy between equation \eqref{PT} and the reaction-diffusion equation $u_t=u_{xx}+f(u)$, where the existence of travelling wave is well-known since the pioneering work of \cite{KPP}. Using comparison principles, in \cite{chen1997existence}, Chen showed the existence of traveling fronts that connect $0$ and $1$ for a very general class of nonlocal evolution equations of the form $u_t(x,t) = \mathcal{A}[u(\cdot,t)](x)$, with $\mathcal{A}$ a nonlinear operator that satisfies various assumptions, among which the most important is the comparison principle: if $u_t \ge \mathcal{A}[u],v_t \le \mathcal{A}[v]$ and $u(\cdot,0) \ge v(\cdot,0)$ but not equal to each other, then $u(\cdot,t)>v(\cdot,t)$ for all $t>0$. Chen constructed the solution by first choosing an appropriate initial data, then evolve it according to the equation, and show that the function of the form $u(\cdot+\xi(t),t)$ will converge to the profile of a traveling wave as $t\to \infty$, where $\xi(t)$ is chosen so that $u(\xi(t),t)=1/2$.


Motivated by \cite{pulseNLFHN}, where the authors studied a nonlocal system of the form 
\begin{subequations}
\begin{eqnarray}
u_t(x,t) &=& -u(x,t) + \int_{\R} \K(x-y)u(y,t)dy+f(u(x,t))-v(x,t),\\
v_t(x,t) &=& \eps(u(x,t)-\gamma v(x,t)),
\end{eqnarray}
\end{subequations}
and proved the existence of travelling front solutions using an extension of the singular-perturbation method. They constructed the slow manifold and the singular solution using cut-off functions and implicit-function theorem, replacing the more common geometric dynamical system methods. In the present paper we continue this functional-analytic approach, and focus on the simpler system \eqref{system}. We will use Fourier transform to rewrite equation \eqref{system} in a form that relates it to the ODE $u''=-\mu u+u^2$, and set up a Newton iteration scheme to continue the solution for $\mu>0$ sufficiently small.

\paragraph{Outline.}The remainder of the paper is organized as follows: after introducing the notations we will use in this paper, in section $2$ we prove our main results, in section $3$ we discuss possible generalizations of our result and future directions. The appendix contains an elementary proof of the nondegeneracy of the operator $\partial_{xx}-d^{-1}(1-2u_*)$. 

\paragraph{Notation.}
We shall use the standard Sobolev spaces $W^{k,p}(\R; \R^n)$, or simply $W^{k,p}$ when $n=1$, for $k \ge 0$ and $1\le p \le \infty$
\[
W^{k,p}(\R;\R^n) := \{ u \in L^p(\R;\R^n): \partial_xu \in L^p(\R;\R^n), 1\le \alpha \le k \},
\]
with norm
\[
\|u\|_{W^{k,p}(\R;\R^n)}=
\begin{cases}
\left(\sum_{1\le \alpha\le k} \|\partial_xu \|_{L^p(\R;\R^n)}\right)^{1/p}, \text{ for }1\le p<\infty \\
\max_{1\le \alpha\le k}\|\partial_x u\|_{L^\infty(\R;\R^n)}, \text{ for }p=\infty.
\end{cases}
\]
We use $H^k(\R;\R^n)$ to denote the space $W^{k,2}(\R;\R^n)$, and $H^k_e(\R;\R^n)$ the subspace of $H^k(\R;\R^n)$ which consists of even functions in $H^k(\R;\R^n)$. We will also use $\mathscr{C}^k(\R;\R^n)$ to denote the space of $k-$times continuously differentiable functions for $k=0,1,\ldots,\infty$.

Finally, we use the usual Fourier transform on $\R$, 
\[
\widehat{f} (\ell)= \int_{\R} f(x)e^{-2\pi i x\ell}dx
\]
 for a Schwartz function $f$. 




\section{Hypothesis and the main result}

We are interested in nontrivial solutions bifurcating from the homogeneous state $U = 0$ as one increases the parameter $\mu$. Before stating the main result, we introduce our hypothesis.

\paragraph{Assumptions on the linear part.}We state our first main hypothesis.
\begin{Hypothesis}{(H1).}Let $I_n$ denote the identity matrix of size $n$. We make the following assumptions on $\K$:
\item (i)  The entries of $\K$ are even, $\K_{i,j}(x)=\K_{i,j}(-x)$, belongs to $W^{2,1}$, and satisfy the weight assumption $\int_{\mathbb{R}} x^{4}|\K_{i,j}(x)| dx<\infty$ for $1\le i,j\le n$.

\item [(ii)] We define the characteristic equation $\mathcal{D}(\ell)$ as
\[\mathcal{D}(\ell) := \det(I_n+\widehat{\K}(\ell)),\hspace{0.1in} \textit{ for } \ell \in \R.\] We then make the following assumptions on the characteristic equation.
\begin{itemize}
\item $\mathcal{D}(0)= 0$ with $\mathcal{D}''(0) := D \neq 0$;
\item $\mathcal{D}(\ell) \neq 0$ for all $\ell \neq 0$.
\end{itemize}
\end{Hypothesis}


By $(i)$ and properties of the Fourier transform, entries of $\widehat{\K}$, and hence $\mathcal{D}$ are even, $\mathscr{C}^4$ functions in $\ell$, which justifies the validity of taking the second derivative of $\mathcal{D}$ in $(ii)$. From $(ii)$ we conclude that $I_n+\widehat{\K}(\ell)$ is invertible for all $\ell$ except $0$. At $\ell = 0$, it readily follows from the above that the kernel of $I_n + \widehat{\K}(0)$ is one dimensional. Therefore there exist $\mathcal{E}_0\in \R^n$ and $\mathcal{E}_0^* \in \R^n$ such that
\[
\widehat{\mathcal{T}}(0)\mathcal{E}_0 = \mathcal{E}_0+\widehat{\K}(0) \mathcal{E}_0 = 0,  \hspace{0.1in} \widehat{\mathcal{T}}(0)^T\mathcal{E}_0^* = \mathcal{E}_0^*+\widehat{\K}(0)^T \mathcal{E}_0^* =0, \hspace{0.1in} \langle \mathcal{E}_0, \mathcal{E}_0\rangle = \langle \mathcal{E}_0^*, \mathcal{E}_0^*\rangle = 1,
\]
where $\widehat{\mathcal{T}}(0)^T$ is the transpose of $\widehat{\mathcal{T}}(0)$, and $\langle \cdot,\cdot \rangle$ denotes the standard inner product on $\R^n$ given by
\[
\langle u,v\rangle = \sum_{i=1}^{n} u_iv_i, \text{ for any }u=(u_i)_{i=1}^n\in \R^n, \text{ and }v=(v_i)_{i=1}^n \in \R^n.
\]

\paragraph{Assumptions on the nonlinear part.}

We state our second main hypothesis.
\begin{Hypothesis} {(H2).} We assume $\Nl=\Nl(U;\mu):\R^n \times \R^+ \to \R^n $ is in $\mathscr{C}^{\infty}(\R^n \times \R^+; \R^n)$. That is, the nonlinear function $\Nl$ is a smooth function in its arguments. Moreover, we require
\begin{enumerate}
\item $\Nl(0;\mu) = 0$ for all $\mu$.

\item The derivative of $\Nl$ satisfies the generic condition 
%\left(S_{cc}^\infty+ \widehat{J}_{cc}(0)\right)\frac{\partial^2}{\partial \mu \partial v_c}\tilde{\Nl}_c(0,0;0)+\left(S_{ch}^\infty+ \widehat{J}_{ch}(0)\right)\frac{\partial^2}{\partial \mu \partial v_c}\tilde{\Nl}_h(0,0;0) \neq 0,
\begin{eqnarray}
\alpha := \langle D_{\mu,U} \Nl(0;0)\mathcal{E}_0, \mathcal{E}_0^*\rangle > 0
 \label{muvCoe}, \\ 
\beta := \langle D_{U,U} \Nl(0;0)[\mathcal{E}_0,\mathcal{E}_0]\mathcal{E}_0, \mathcal{E}_0^*\rangle \neq 0  \label{QuadCoe}.
\end{eqnarray}

%\left(S_{cc}^\infty+ \widehat{J}_{cc}(0)\right)\frac{\partial^2}{\partial v_c^2}\tilde{\Nl}_c(0,0;0)+\left(S_{ch}^\infty+ \widehat{J}_{ch}(0)\right)\frac{\partial^2}{\partial v_c^2}\tilde{\Nl}_h(0,0;0) \neq 0,
%where $S(0)$ is the matrix $S(\ell)$ constructed in lemma \ref{Lem1} evaluated at $\ell = 0$.

\end{enumerate}

\end{Hypothesis}

Note that by the above hypothesis, using Sobolev embedding, the superposition operator $U(\cdot) \mapsto \Nl(U(\cdot);\mu)$ maps $H^2(\R;\R^n)$ into $H^2(\R;\R^n)$ and is smooth. Part $(ii)$ of our hypothesis is analogous to the generic transcritical bifurcation assumption (see chapter 6.6 of \cite{chow1982methods}), a typical example is $\Nl(U;\mu)=\mu U-U^2$ when $n=1$.


We can now state our main result.
 
\begin{theorem}\label{MainRes} Assume hypothesis (H1) and (H2). Then there exist a positive constant 
$\mu_0$ sufficiently small, such that \eqref{system} has a family 
of nontrivial solutions $U_*=U_*(\cdot;\mu) \in H_e^2(\R;\R^n)$ 
parametrized by $\mu\in (0,\mu_0) $, with the expression 
\begin{equation}\label{expan}
U_*(x;\mu) =  v_c(x;\mu)\mathcal{E}_0 + v_{\perp}(x;\mu)
\end{equation}
Here, $v_{\perp}$ takes value in the complement of $\mathcal{E}_0$, and satisfies $\|v_{\perp}\|_{H^2(\R;\R^{n-1})} = \rmO(\mu^2)$ as $\mu \to 0$. $v_c$ is a scalar function of the form 
$v_c(x; \mu)=-\frac{1}{\beta} (\alpha\mu)\left[ v_*(\sqrt{\alpha\mu}x)+w(\sqrt{\alpha\mu}x;\sqrt{\alpha\mu}) \right]$. Here $w(\cdot;\sqrt{\alpha\mu}) \in H^2_e$ is a corrector function with $\|w(\cdot;\sqrt{\alpha\mu})\|_{H^2} \to 0$ as $\mu \to 0$, and $v_*(\cdot)$ is the unique solution to the equation $dv^{''} -u+u^2 =0$ where $d = \frac{1}{2}\langle \widehat{\K}^{''}(0)\mathcal{E}_0,\mathcal{E}_0^*\rangle$, with the condition 
$v_*(0)>0$ and $v_*(\pm\infty)=\lim_{\xi \to \infty} v(\xi)=0$.

\end{theorem}
\subsection{Normal form transformation}
We transform the operator $I_n+\K \ast$ to a ``normal form'' via the following lemma.


\begin{lemma}\label{Lem1} There exist invertible $m \times m$ matrices $P, Q$, and a $\ell$-dependent matrix $S(\ell)$, such that for all $\ell \in \R$, we have
\[
S(\ell)P[I_n+\widehat{\K}(\ell)]Q = \diag\{m(\ell),I_{n-1} \},
\]
where $m(\ell) = \dfrac{d\ell^2}{1+\ell^2}$ and $d = D\det(PQ) \neq 0$. Moreover, the entries of the matrix $S(\ell)$ are $\mathscr{C}^2(\R)$, and $S(\ell)$ decomposed as $S(\ell)=S_r(\ell)+S^\infty$, where $S^\infty$ is a constant matrix, and the entries of $S_r(\ell)$ are integrable.  
\end{lemma}
\begin{Proof}Set $\mathcal{T}(\ell) := I_n+\widehat{\K}(\ell)$, we divide our construction in 2 steps.

\paragraph{Step 1.}
 When $\ell = 0$, the rank of $\mathcal{T}(0)=I_n+\widehat{\K}(0)$ is equal to $n-1$ since it has a one dimensional kernel spanned by $\mathcal{E}_0$, it is standard (see for example \cite{roman2007advanced}) that there exist invertible matrices $P$ and $Q$ such that
\[
P\mathcal{T}(0)Q = \diag\{0,I_{n-1}\}.
\]
In order to do computations later, we include here a construction of the matrices $P$ and $Q$. Let $\mathcal{E}_1,\ldots,\mathcal{E}_{n-1}$ be a set of vectors which span the complement of $\ker \mathcal{T}(0)$. Set $\mathcal{F}_i = \mathcal{T}(0)\mathcal{E}_i$ for $i=1,\ldots,n-1$. Define $Q$ to be the matrix whose column vectors are $\mathcal{E}_0,\ldots,\mathcal{E}_{n-1}$ (with respect to the standard basis in $\R^n$), and $P$ so that the columns of $P^{-1}$ are given by $\mathcal{E}_0^*,\mathcal{F}_1,\ldots,\mathcal{F}_{n-1}$. By the Fredholm alternative, we know that $\mathcal{E}_0^*$ is orthogonal to the subspace spanned by $\mathcal{F}_i$, that is $\langle\mathcal{E}_0^*,\mathcal{F}_i\rangle = 0$ for $i=1,\ldots,n-1$. 

 Then we have 
\[
P\mathcal{T}(0)Q = P\mathcal{T}(0)[\mathcal{E}_0,\ldots,\mathcal{E}_{n-1}] = P[0,\mathcal{F}_1,\ldots,\mathcal{F}_{n-1}]=\diag\{0,I_{n-1}\},
\]
which is the desired diagonalization.

Recall that $\widehat{\K}(\ell)_{i,j}$ are $\mathscr{C}^4$ functions in $\ell$, since $\K$ is even, by Taylor's theorem, we have the expansion
\[
P\mathcal{T}(\ell)Q = P[I_n+\widehat{\K}(0)]Q+\frac{1}{2}P[I_n+\widehat{\K}''(0)]Q \ell^2 + \rmO(\ell^4) ,
\]
at $\ell =0 $. It then follows that near $\ell = 0$, we have
\[
P\mathcal{T}(\ell)Q = \begin{pmatrix}
A(\ell)& B(\ell)\\
C(\ell)& D(\ell) 
\end{pmatrix},
\]
where $A(\ell) = d\ell^2+\rmO(\ell^4)$ for some constant $d$, $B(\ell), C(\ell)$ are $n-1$ sized row, column vectors respectively, both satisfies $B(\ell),C(\ell) =\rmO(\ell^2)$, and $D(\ell) =I_{n-1}+\rmO(\ell^2)$. Note that $d = \frac{1}{2}\langle \widehat{K}''(0)\mathcal{E}_0,\mathcal{E}_0^*\rangle$ verifying the claim in the statement of Theorem \ref{MainRes}.
%\left(
%\begin{array}{c|c}
 % d\ell^2 +\rmO(\ell^4) & \cdots \rmO(\ell^2) \cdots\\ \hline
 % \vdots & \raisebox{-15pt}{{\large\mbox{{$I_{n-1}+\rmO(\ell^2)$}}}} \\[-4ex]
  %\rmO(\ell^2) & \\[-0.5ex]
%  \vdots &
%\end{array}
%\right)

Next we claim that $d\neq 0$, recall that the determinant of a block diagonal matrix of the form $\begin{pmatrix}
A&B\\
C&D
\end{pmatrix}$ is given by the formula $\det(D)\det(A-BD^{-1}C)$ provided $D$ is invertible, which is the case for $\ell$ sufficiently close to $0$. We compute
\begin{align*}
\det P\mathcal{T}(\ell)Q &=\det (D(\ell))\det[A(\ell)-B(\ell)D(\ell)^{-1}C(\ell)] \\
& = \det(D(\ell)) (A(\ell)- B(\ell)D(\ell)^{-1}C(\ell)\\
& =(1+\rmO(\ell^2)) (d\ell^2 + \rmO(\ell^4)) = d\ell^2 +\rmO(\ell^4).
\end{align*}
By assumption, $\det P\mathcal{T}(\ell)Q = \det(PQ)\det(\mathcal{T}(\ell))=\det(PQ)(D\ell^2+\rmO(\ell^4))$ as $\ell \to 0$, hence $d = D\det(PQ)$, which is nonzero as claimed.


\paragraph{Step 2.} Set $
H(\ell) = \diag\{\frac{1+\ell^2}{d\ell^2}, I_{n-1}\}$ for $\ell \neq 0$. First we let $\tilde{S}(\ell) = P\mathcal{T}(\ell)Q H(\ell) $
for $\ell \neq 0$. For $\ell = 0$, we compute
\begin{align*}
\lim_{\ell \to 0} P\mathcal{T}(\ell)QH(\ell) &=\lim_{\ell \to 0} \left(
\begin{array}{c|c}
  d\ell^2+\rmO(\ell^4) &  \cdots \rmO(\ell^2)\cdots \\ \hline
  \vdots & \raisebox{-10pt}{{\Large\mbox{{$I_{n-1}+\rmO(\ell^2)$}}}} \\[-4ex]
  \rmO(\ell^2) & \\[-0.5ex]
  \vdots &
\end{array}\right)\left(
\begin{array}{c|c}
  \frac{1+\ell^2}{d\ell^2} & 0 \cdots 0 \\ \hline
  0 & \raisebox{-10pt}{{\Large\mbox{{$I_{n-1}$}}}} \\[-4ex]
  \vdots & \\[-0.5ex]
  0 &
\end{array}\right)\\
&=\lim_{\ell \to 0}\left(
\begin{array}{c|c}
  1+\ell^2+\rmO(\ell^4) &  \cdots \rmO(\ell^2)\cdots \\ \hline
  \vdots & \raisebox{-10pt}{{\Large\mbox{{$I_{n-1}+\rmO(\ell^2)$}}}} \\[-4ex]
  \ast & \\[-0.5ex]
  \vdots &
\end{array}\right)=\left(
\begin{array}{c|c}
  1 & 0 \cdots 0 \\ \hline
  * & \raisebox{-10pt}{{\Large\mbox{{$I_{n-1}$}}}} \\[-4ex]
  \vdots & \\[-0.5ex]
  * &
\end{array}\right),
\end{align*} 
which is an invertible matrix, and we define it to be $\tilde{S}(0)$.

Next, we define $S(\ell)$ so that $S(\ell) = \tilde{S}(\ell)^{-1}$. For $\ell \neq 0$, $\tilde{S}(\ell) \in \mathscr{C}^4$ since $\mathcal{T} \in \mathscr{C}^4$. Near $\ell = 0$, we can write $P\mathcal{T}(\ell)Q = \begin{pmatrix}
\ell^2\tilde{A}(\ell)&B(\ell)\\
\ell^2\tilde{C}(\ell)&D(\ell)
\end{pmatrix}$ such that $\tilde{A},\tilde{C}$ are of class $\mathscr{C}^2$ by Taylor's theorem. As a consequence $\tilde{S}(\ell)= PT(\ell)QH(\ell)$ is $\mathscr{C}^2$ near $\ell = 0$. Since $\tilde{S}^{-1}(0)$ is an invertible matrix, by the implicit function theorem, we conclude that the entries of $S$ are $\mathscr{C}^2$ functions in $\ell$.
 
Hence, if we apply $S(\ell)$ to $P\mathcal{T}(\ell)Q$ from the left, it then yields
 \[
 S(\ell)P\mathcal{T}(\ell)Q = H(\ell)^{-1}[P\mathcal{T}(\ell)Q]^{-1}P\mathcal{T}(\ell)Q = M(\ell)^{-1}=\diag\{m(\ell),I_{n-1}\},
 \]
 which is as stated in the lemma.
 The entries of $S(\ell)$ defined here are continuous for all $\ell \in \R$,  also, by definition, for $\ell\neq 0$, we have $S(\ell) = H(\ell)^{-1}[P\mathcal{T}(\ell)Q]^{-1} = H(\ell)^{-1}Q^{-1}\mathcal{T}(\ell)^{-1}P^{-1}$.
 
By (H1), the entries of $\K$ are of class $W^{2,1}$, hence the Fourier transform $\widehat{\K}$ satisfies $\ell^2|\widehat{\K}_{i,j}(\ell)| \to 0$ as $\ell \to 0$ for $1\le i,j\le n$. Therefore $\ell^2(\mathcal{T}(\ell)-I_n)=\ell^2(\widehat{\K}(\ell))$ converges to the $0$ matrix as $\ell \to 
\infty$. If we set $S^\infty := \diag\{d,I_{n-1}\}Q^{-1}P^{-1}$, which is an invertible matrix as $d\neq 0$, it then follows that $\ell^2(S(\ell)-S^\infty)\to 0$ as $\ell \to \infty$. 

Therefore, set $S_r(\ell) = S(\ell)-S^{\infty}$. We have defined $S(\ell)$ so it is continuous at $\ell=0$, and it decays faster than $\ell^{-2}$ as $\ell \to \infty$. We conclude that the entries of $S_r(\ell)$ are integrable. 

Finally, since $S(\ell)$ is invertible at $\ell =0$, and converges to the invertible matrix $S^\infty$ at $\ell =\infty$, we conclude that  $S(\ell)$ is invertible for all $\ell$, with uniform bounds on its inverse. This concludes the proof.
 \end{Proof}


Using Fourier inversion, $S_r(\ell)$ is induced by a convolution with a matrix kernel $J$ whose entries are integrable functions, that is, for a $\R^n$-valued function $U(x)$ defined for $x\in \R$, we have $\widehat{J \ast U} (\ell) = S_r(\ell)\widehat{U}(\ell)$,

Therefore we set $V(x)=Q^{-1}U(x)$, with standard coordinates $V(x)=(v_c(x),v_2(x),\ldots,v_{n}(x))^T$. Define the multiplier operator $M$ through $\widehat{Mv}(\ell) = m(\ell)\widehat{v}(\ell)$ for $v\in L^2$. We then precondition \eqref{system} with the operator $L := S^{\infty}+J \ast$ whose symbol equals $S(\ell)$, to obtain a new equation
\begin{equation}\label{TranEq}
\diag(M, I_{n-1})V=(S^\infty+J \ast )P\Nl(QV;\mu).
\end{equation}

In the next subsection we shall further simplify \eqref{TranEq} by rescaling the variables, so that its behavior as $\mu \to 0$ will be revealed.

\subsection{Change of coordinates and rescaling }
% If $\mathcal{E}$ denotes the vector which spans the kernel of $\mathcal{T}(0)=I_n+\widehat{\K}(0)$, let $\mathcal{E}_2,\ldots,\mathcal{E}_{n-1}$ be a set of basis which spans the complement of $\ker \mathcal{T}(0)$, set $\mathcal{F}_i = \mathcal{T}(0)\mathcal{E}_i$, choose $\mathcal{E}_1^*$ such that $\langle \mathcal{E}_1^*,\mathcal{E}_1^*\rangle=1$ and $\langle \mathcal{E}_1^*, \mathcal{F}_i\rangle=0$ where $\langle \cdot,\cdot \rangle$ is the usual scalar product in $\R^n$.




%\paragraph{Derivative of $\tilde{\Nl}$}
%then we have $\frac{\partial^2}{\partial \mu \partial v_c}\tilde{\Nl}_c(0;0)$ is given by $\langle \mathcal{E}_1^*, D_{\mu,U} \Nl(0;0) \mathcal{E}_1\rangle$, and the $n-1$ components of $\frac{\partial^2}{\partial\mu \partial v_c}\tilde{\Nl}_h(0;0)$ is given by $\langle\mathcal{F}_i, D_{\mu,U} \Nl(0;0)\mathcal{E}_1\rangle$ for $i=2,\ldots,n-1$. Similarly, the second derivative in $U$ can be calculated as: $\frac{\partial^2}{\partial v_c^2}\tilde{\Nl_c}(0;0)=\langle\mathcal{E}_1^*, D_{U,U}\Nl(0;0)\mathcal{E}_1\rangle$ and $\frac{\partial^2}{\partial v_c^2}\tilde{\Nl_h}(0;0)=\langle\mathcal{F}_i, D_{U,U}\Nl(0;0)[\mathcal{E}_1,\mathcal{E}_1]\rangle$ for $i=2,\ldots,n-1$.


\paragraph{Transformed equations.} 
Write $v_h = (v_2,\ldots,v_{n})^T$ in the standard coordinate in $\R^{n-1}$. Set $\tilde{\Nl}(V;\mu) :=P\Nl(QV;\mu)$. Then with respect to the standard basis in $\R^n$, we define $\tilde{\Nl}_c(V;\mu)$ to be the first component of the nonlinearity $\tilde{\Nl}(V;\mu)$ and $\tilde{\Nl}_h(V;\mu)$ to be the remaining $n-1$ components of $\tilde{\Nl}(V;\mu)$. Then, also with respect to the usual basis, we write the matrix $S^\infty$ and the matrix convolution kernel $J$ in components
\[
S^{\infty} = \begin{pmatrix}
S^{\infty}_{cc} & S^{\infty}_{ch}\\
S^{\infty}_{hc} & S^{\infty}_{hh} 
\end{pmatrix}, \hspace{0.2in}
J\ast = \begin{pmatrix}
J_{cc}\ast & J_{ch}\ast \\
J_{hc}\ast & J_{hh}\ast 
\end{pmatrix}, \hspace{0.2in}
L  = S^\infty+J\ast = \begin{pmatrix}
L_{cc}\ast & L_{ch}\ast \\
L_{hc}\ast & L_{hh}\ast 
\end{pmatrix},
\]
where terms with subscript $cc$ denote a scalar, terms with subscript $ch$ a $(n-1)$ dimensional row vector, terms with subscript $hc$ a $(n-1)$ dimensional column vector, terms with subscript $hh$ a $(n-1)\times (n-1)$ matrix.

In this notation, system \eqref{system} becomes
\begin{align}
M v_c + L_{cc}\tilde{\Nl}_c(v_c,v_h;\mu) + L_{ch}\tilde{\Nl}_h(v_c,v_h;\mu)= 0\label{exeqnu0},\\
v_h +  L_{hc}\tilde{\Nl}_c(v_c,v_h;\mu) + L_{hh}\tilde{\Nl}_h(v_c,v_h;\mu) = 0 \label{exeqnuh}.
\end{align}


By part $(i)$ of the hypothesis (H2), we may write the Taylor jet of $\tilde{\Nl}_j$ with $j=c,h$ as
\begin{align*}
\tilde{\Nl}_j(v_c,v_h;\mu) &=\left( a^j_{101} \mu v_c+a^j_{011}\mu v_h+a^j_{110}v_cv_h + a^j_{200}v_c^2+a^j_{020}[v_h,v_h] \right)+ \Rm_j(v_c,v_h;\mu)\\
&:= \B_j(v_c,v_h;\mu)+\Rm_j(v_c,v_h;\mu),
\end{align*}
where $a^j_{lmn} = \frac{\partial^2 \tilde{\Nl}_j}{\partial v_c^l \partial v_h^m \partial \mu^n}(0,0;0), l+m+n=2, 0\le l,m,n\le 2$ are the partial derivatives of order $2$ at $(v_c,v_h;\mu)=(0,0;0)$, and the remainder $R_j$ satisfies 
\begin{equation}\label{odR}
|R_j(v_c,v_h;\mu)| = \rmO(v_c^2|v_h|,v_c|v_h|^2,v_c^3,|v_h|^3,\mu v_c^2, \mu |v_h|^2,\mu v_c|v_h|, \mu^2v_c, \mu^2|v_h|),
\end{equation} 
as $(v_c,v_h;\mu) \to (0,0;0)$.

We are in particular interested in the coefficients of the term $\mu v_c$ and $v_c^2$. In equation \eqref{exeqnu0}, the coefficient of $\mu v_c$ is given by $L_{cc}a_{101}^c+L_{ch}a_{101}^h$, and the coefficient of $v_c^2$ is given by $L_{cc}a_{200}^c+L_{ch}a_{200}^h$. Using hypothesis (H2), we claim that
\[
\alpha = a_{101}^c =\widehat{L}_{cc}(0)a_{101}^c+\widehat{L}_{ch}(0)a_{101}^h , \hspace{0.1in}\text{ and }\hspace{0.1in}
\beta =a_{200}^c =\widehat{L}_{cc}(0)a_{200}^c+\widehat{L}_{ch}(0)a_{200}^h .
\]

Indeed, to verify the first assertion, recall that by definition we have $\widehat{L}(0) = S(0)$. In particular, 
$\widehat{L}_{cc}(0)=1$ and $\widehat{L}_{ch}(0)=(0,\ldots,0)$, thus verifying the second equality $a_{101}^c=\widehat{L}_cc(0)a_{101}^c+\widehat{L}_{ch}a_{101}^h$. 
To compute the coefficient of $\mu v_c$ in terms of the 
original function $\Nl$, let $e_1$ denote the standard coordinate vector $(1,0,\ldots,0)^T$, then the derivative 
$a_{101}^c=\dfrac{\partial^2}{\partial \mu \partial v_c}  \tilde{\Nl}_c(0,0;0)$ is given by
\[
\langle D_{\mu,V}\tilde{\Nl}(0;0)e_1,e_1\rangle=\langle D_{\mu,U} P\Nl(0;0)Q e_1,e_1\rangle = \langle PD_{\mu,U}\Nl(0;0) \mathcal{E}_0,e_1\rangle  = \langle D_{\mu,U}\Nl(0;0)\mathcal{E}_0,\mathcal{E}_0^*\rangle = \alpha,
\]
which verifies the first equality $\alpha=a_{101}^c$. The computations for $\beta$ is similar.

In the next paragraph, we shall make a series of rescalings to simplify the equation.

\paragraph{Rescaling.} Define new variables $\tilde{v}_c,\tilde{v}_h, \tilde{\mu}$ through 
\[
\mu=\frac{1}{\alpha}\tilde{\mu}, \hspace{0.1in} v_c(\cdot)=\frac{-1}{\beta}\tilde{\mu}\tilde{v}_c(\sqrt{\tilde{\mu}} \cdot), \hspace{0.1in} v_h(\cdot)=\tilde{\mu} \tilde{v}_h(\sqrt{\tilde{\mu}} \cdot),
\]
since we assumed $\alpha >0$, $\tilde{\mu}$ is positive, and we shall write $\eps := \sqrt{\tilde{\mu}}$.

We substitute these variables in equation \eqref{exeqnu0} and \eqref{exeqnuh}, divide the first equation by $(-1/\beta)\eps^4$, the second by $\eps^2$, and then obtain
\begin{align}
\eps^{-2}M^\eps \tilde{v}_c + L_{cc}^{\eps}(\tilde{\B}_c+\eps^{-4}\tilde{\Rm}_c) +L_{ch}^{\eps}(\tilde{\B}_h+\eps^{-4}\tilde{\Rm}_h) = 0,\label{rseqnu0}\\
\tilde{v}_h + L_{hc}^{\eps}(\eps^2\tilde{\B}_c+\eps^{-2}\tilde{\Rm}_c) +L_{hh}^{\eps}(\eps^2\tilde{\B}_h+\eps^{-2}\tilde{\Rm}_h)= 0. \label{rseqnuh}
\end{align}


Here the nonlinear terms $\tilde{\B_j}, \tilde{\Rm_j}$ are defined through 
\begin{align*}
\tilde{\B}_j(\tilde{v}_c,\tilde{v}_h)&=\left(\frac{a^j_{101}}{\alpha}  \tilde{v}_c+\frac{a^j_{011}}{\alpha} \tilde{v}_h+a^j_{110}\tilde{v}_c\tilde{v}_h + \frac{a^j_{200}}{-\beta}\tilde{ v}_c^2+a^j_{020}(-\beta)\tilde{v}_h^2 \right),\\
\tilde{R}_j ( \tilde{v}_c,\tilde{v}_h;\eps)&=\Rm_j \left(\frac{\eps^2\tilde{v}_c}{-\beta},\eps^2\tilde{v}_h;\frac{\eps^2}{\alpha}\right).
\end{align*}
 
Note that, by an application of the Taylor theorem with remainder, the terms $\tilde{\Rm}_j$, together with its derivatives, are of order $\rmO(\mu^3) =\rmO( \eps^6)$ for $\tilde{v}_c,\tilde{v}_h$ bounded as $\eps\to 0$. 

The linear term $M^\eps $ is the operator with symbol $m(\eps \ell)$,  and $L_j^\eps = S^\infty_j+ J_j^\eps \ast$ denotes the rescaled version of $L_j$, where $J_j^\eps(\cdot)$ is the rescaled convolution kernel
\[ 
J_j^\eps(\cdot) = \eps^{-1}J_j(\eps^{-1}\cdot).
\]
Note that, since the $J_j$ are integrable, $J_j^\eps \ast u$ will converge to $\left(\int_\R J_j\right) u= \widehat{J}_j(0) u$ as $\eps \to 0$ in $L^p(\R)$ for any $u \in L^p(\R)$ with $p \in [1,\infty)$. Hence, in the limit $\eps \to 0$, $L^\eps_j$ converges to $\widehat{L}(0) = S(0)$. In particular, the coefficient of the term $\tilde{v}_c$ equals $ a_{101}^c/\alpha=1$, and the coefficient of $\tilde{v}_c^2$ equals $a_{200}^h/(-\beta)=\beta/(-\beta)=-1$ by the computation we have done in the previous step. As a consequence, we have 
\[
\tilde{B}_c(\tilde{v}_c,\tilde{v}_h) = \tilde{v}_c-\tilde{v}_c^2 + \rmO(|\tilde{v}_h|+|v_h|^2+v_c|v_h|),
\] as $(v_c,v_h) \to (0,0)$. 
To further ease notations, we drop the tildes, and still use $v_j,\B_j,\Rm_j$ for the same variables after the rescaling. We will prove our main result in the next subsection.



\subsection{Lyapunov-Schmidt reduction and proof of the main result}
In this subsection, we first solve \eqref{rseqnuh} to obtain $v_h$ as a function of $v_c$ by a fixed point argument. We then substitute this function back into equation \eqref{rseqnu0} to obtain a scalar equation for $v_c$ and $\eps$, which will be solved again using a fixed point argument.

We write the left hand side of \eqref{rseqnuh} as $\G(v_h; v_c,\eps)$, so that
\[
\G(v;u,\eps) = v+L_{hc}^{\eps}\left( \eps^2\B_c+\eps^{-2}\Rm_c \right)
+L_{hh}^{\eps}\left( \eps^2\B_h+\eps^{-2}\Rm_h \right). \]
In other words we are treating $v_c$ as an additional (Banach space-valued) parameter, and we have the following lemma.

\begin{lemma}\label{Lemuh} Fix $r>0$ not necessarily small, and let $B_r$ denote the ball centered at $0$ with radius $r$ in $H^2$, there then exist $\eps_0>0$ sufficiently small and a map $\psi(u,\eps): B_r \times (-\eps_0,\eps_0) \to H^2(\R;\R^{n-1})$ such that $v = \psi(u, \eps)$ solves $\G(\psi(u,\eps);u,\eps) = 0$. Moreover, the map $u \mapsto \psi(u,\eps)$ is smooth for $u\in B_r$, and we have 
\[
\|\psi(u,\eps)\|_{H^2(\R,\R^{n-1})} = \rmO(\eps^2), \hspace{0.1in}\|D_u\psi(u,\eps)\|_{H^2 \to H^2(\R;\R^{n-1})} = \rmO(\eps^2),
\] as 
$\eps \to 0$, uniformly for $u\in B_r$ where $D_u\psi(u,\eps)$ denotes the Frechet derivative of $\psi$ with respect to $u$ at the point $(u,\eps)$.  \end{lemma}
\begin{Proof}  Recall that by definition $\B_j(u,v)=\left(a^j_{101} u+a^j_{011} v+a^j_{110}uv + a^j_{200}u^2+a^j_{020}[v,v] \right)$ and $\Rm_j = \Nl_j-\eps^4\B_j$. Using the fact that $H^2$ is a multiplication algebra, it follows that for $u \in H^2, v \in H^2(\R;\R^{n-1})$ the estimates
\begin{align*}
&\|\B_c(u,v)\|_{H^2},\|\B_h(u,v)\|_{H^2(\R;\R^{n-1})} \\
 &\le C\left(\|u\|_{H^2}+\|v\|_{H^2(\R;\R^{n-1})}+\|u\|_{H^2}\|v\|_{H^2(\R;\R^{n-1})} +\|u\|^2+\|v\|^2_{H^2(\R;\R^{n-1})}\right)
\end{align*}
hold for some constant $C$. We also have
\[
 \|\Rm_c(u,v;\eps)\|_{H^2},  \|\Rm_h(u,v;\eps)\|_{H^2(\R;\R^{n-1})} = \rmO(\eps^6), 
\]
and
\[
\|[D_v \Rm_c(u,v;\eps) w_1\|_{H^2}, \|[D_v \Rm_h(u,v;\eps) w_2\|_{H^2(\R;\R^{n-1})} = \rmO(\eps^6)
\]
for any $w_1 \in H^2, w_2 \in H^2(\R;\R^{n-1})$ with $\|w_1\|_{H^2}= \|w_2\|_{H^2(\R;\R^{n-1})}=1$ as $\eps \to 0$ from \eqref{odR}.


Next, the norm of the linear operators $L^\eps_{cc} : H^2 \to H^2, L^\eps_{ch}:H^2(\R;\R^{n-1}) \to H^2, L^\eps_{hc}:H^2 \to H^2(\R;\R^{n-1}),$ and $L^\eps_{hh}:H^2(\R;\R^{n-1}) \to H^2(\R;\R^{n-1})$ are given by the supremum of their symbols $\widehat{L}^\epsilon_j (\ell)=\widehat{L}(\epsilon\ell)$. We conclude that these operators are bounded with uniform bounds in $\eps$.

We will now solve $\G(v;u,\eps)=0$ using a Newton iteration scheme. For $u \in B_r$ and $\eps_0$ small, we claim the following properties hold for $\G$:
\begin{enumerate}
\item $\|\G(0;u,\eps)\|_{[H^2(\R;\R^{n-1})} = O(\eps^2),$ uniformly in $u\in B_r$ and $|\eps| < \eps_0$.
\item $\G$ is smooth in $v$, and $D_v \G(0; u, \eps):H^2(\R;\R^{n-1}) \to H^2(\R;\R^{n-1})$ is invertible with uniform bounds on the inverse for $|\eps|<\eps_0$ and $u \in B_r$. 
\end{enumerate}

For $(i)$, we compute 
\begin{align*}
\|\G(0,u;\eps)\|_{H^2(\R;\R^{n-1})} &\le \eps^2\left(\| L_{hc}^\eps)\|_{H^2(\R;\R^{n-1}) \to H^2}+\| L_{hh}^\eps)\|_{H^2(\R;\R^{n-1}) \to H^2(\R;\R^{n-1})}\right)(C (\|u\|_{H^2}^2+\|u\|_{H^2})\\
&+\eps^{-4}(\|\Rm_c(u,0;\eps)\|_{H^2}+\|\Rm_h(u,0;\eps)\|_{H^2(\R;\R^{n-1})}) = \rmO(\eps^2),
\end{align*}
uniformly for $u\in B_r$.

For $(ii)$, $\G$ is smooth in $v$ because the nonlinearity $\Nl$ is smooth in its arguments, and $L_j^\eps$ are bounded linear operators. We compute the Frechet derivative of $\G$ to obtain
\[ 
D_v\G(0;u,\eps) w = w + \eps^2((L_{hc}^\eps a_{011}^c+L_{hh}^\eps a_{011}^h)w+(a_{110}^c+a_{110}^h)u w)+\eps^{-2}(D_v\Rm_j +D_v\Rm_h)w.
\] 

 We see $D_v\G(0; u, \eps)$ is an $\rmO(\eps^2)$ perturbation of the identity as an operator on $H^2(\R;\R^{n-1})$ uniformly for $u\in B_r$. Thus, if $\eps_0$ is small enough, then for all $\eps$ with $|\eps|<\eps_0$, we have that $D_v\G(0;u,\eps)$ is invertible with uniform bounds in $\eps$.


After establishing these two points, fix $\delta>0$ and $u \in B_r$. Let $X=B_\delta$ denote the closed ball of radius $\delta$ around $0$ in $H^2(\R;\R^{n-1})$, we introduce a map $\cS(\cdot; u,\eps): X \to X$ as follows:
\[
\cS(v; u,\eps) = v - D_v\G(0;u, \eps)^{-1}[\G(v;u,\eps)].
\]
We then find
\[
\|\cS(0;u,\eps) \|_{H^2(\R;\R^{n-1})} \le \|D_v\G(0;u, \eps)^{-1}\| \|\G(0;u, \eps)\|_{H^2(\R;\R^{n-1})} = \rmO(\eps^2).
\]

Also, $D_v\cS(0;u,\eps) = 0$ by definition. We know that $\cS$ is smooth in $v$ by $(ii)$. Therefore, if $\delta$ is small and $v\in X$, it then follows that $\|D_vS(v;u,\eps)\|_{H^2\to H^2} \le C\delta$ for some constant $C$ independent of $\delta$.

Then we start our iteration with $v_0 = 0$, $v_{n+1} = \cS(v_n;u,\eps)$, $n\ge 0$. Suppose by induction $v_k \in X$ for $1\le k \le n$, then
\[
\|v_{n+1}-v_n\|_{H^2(\R;\R^{n-1})} \le C\delta\|v_n-v_{n-1}\|_{H^2(\R;\R^{n-1})},
\]
by the mean value theorem. Therefore
\[
\|v_{n+1}\|_{H^2(\R;\R^{n-1})} \le \frac{C}{1-C\delta}\|v_1-v_0\|_{H^2(\R;\R^{n-1})} = \frac{C}{1-C\delta}\|\cS(0;u,\eps)\|_{H^2(\R;\R^{n-1})}.
\]
This implies that for $\eps$ small and $u \in B_r$, we have $v_{n+1} \in X$, and that $\cS$ is a contraction for $\delta$ sufficiently small. We then apply Banach's fixed point theorem to obtain $v = \psi(u,\eps)$ as a fixed point of $\cS$, so that $\psi(u,\eps)= \cS(\psi(u,\eps);u,\eps)$, and consequently $\G(\psi(u,\eps); u,\eps) = 0$. Note that we obtain the estimate $\|\psi\|_{H^2(\R;\R^{n-1})} = \rmO(\eps^2)$ from the iteration.

The smooth dependence of $\psi(u,\eps)$ on $u$ is a consequence of the uniform contraction theorem: note $\G(v;u,\eps)$, and therefore the map $\cS$ is smooth in $u$ by assumptions on the nonlinearity, by choosing $\eps$ small, the contraction constant for $\cS$ can be chosen uniformly in $u \in B_r$, hence $\psi$ depends smoothly on $u$ as well.

The estimate on the derivative $\|D_u\psi\|$ is obtained by differentiating the equation $0 = \G(\psi(u,\eps);u,\eps)$ in $u$ for $u\in B_r$, we see $D_u\psi = -[D_v\G]^{-1}D_u\G$, but $\| D_u \G (v;u,\eps)\|= \rmO(\eps^2)$ for $u \in B_r$, as it is given by a formula similar to that of $D_v\G$, and $D_vG$ is uniformly invertible for $\eps$ small and $v\in X$, hence $\|D_u\psi\|_{H^2(\R;\R^{n-1})\to H^2(\R;\R^{n-1})} = \rmO(\eps^2)$ as claimed. 
\end{Proof}

\begin{remark} Because of the dependence of the convolution operator $L_{hc}^\eps, L_{hh}^\eps$ on $\eps$ is not smooth at $\eps = 0$, we cannot use the usual implicit function theorem directly to solve the equation $\G(v;u,\eps) = 0$. We adopt the Newton iteration scheme as in \citep{faye2013existence} to circumvent this problem.
\end{remark}

Using lemma \ref{Lemuh}, we substitute $v_h = \psi(v_c,\eps)$ into equation \eqref{rseqnu0}. We obtain the following scalar equation

\begin{equation} \label{1dnl}
0 = \eps^{-2}m^\eps v_c + \sum_{j=c,h}L_{cj}^\eps\left[B_j(v_c,\psi(v_c,\eps))+\eps^{-4}\Rm_j(v_c,\psi(v_c,\eps);\eps)\right].
\end{equation}


It is now crucial to understand the behavior of the operator $M^\eps$ as $\eps \to 0$. Recall that by definition
\[
\widehat{M^\eps v}(\ell) = m(\eps \ell)\widehat{v}(\ell) = \frac{d(\eps\ell)^2}{1+(\eps\ell)^2} \widehat{v}(\ell), 
\]
for any $v\in L^2$. We then define a new operator $\mathcal{M}^\eps$ through 
\[ 
\widehat{\mathcal{M}^\eps v}(\ell) = \frac{m(\eps\ell)}{(\eps\ell)^2}\widehat{v}(\ell)=\frac{d}{1+(\eps\ell)^2} \widehat{v}(\ell). 
\] 
Note that $d/(1+(\eps \ell)^2)$ is a bounded function on $\R$, so $\M^\eps$ maps $L^2$ into itself. 

For $v\in L^2$, $(\M^{\eps})^{-1}$ is defined through
\[
\widehat{(\M^{\eps})^{-1}v} (\ell) = \frac{(\eps\ell)^2}{m(\eps\ell)} \widehat{v}(\ell)= \frac{1+(\eps \ell)^2}{d} \widehat{v}(\ell),
\]
Moreover, for $v \in H^2$, we obtain:
\begin{align*}
\|((\M^\eps)^{-1}-d^{-1})v\|_{L^2} &=\left\| \left(\frac{(\eps\ell)^2(1+(\eps\ell)^2)}{d(\eps\ell)^2}-d^{-1}\right)\widehat{v}(\ell)\right\|_{L^2} \le 
\\
& \le \sup_{\ell} \left|\frac{(\eps\ell)^2}{1+\ell^2}\right| \|(1+\ell^2)\widehat{v}(\ell) \|_{L^2} \le \eps^2 \|v\|_{H^2}.
\end{align*}

Therefore, considered as an operator from $H^2$ to $L^2$, $(\M^\eps)^{-1}$ is well-defined, and $\|(\M^{\eps})^{-1}v - d^{-1}v\|_{L^2} \to 0$ as $\eps \to 0$ for $v \in H^2$. This simple observation is central to identifying the leading-order terms and  we state it as a lemma.


\begin{lemma}\label{estmult}The multiplier operator $(\M^\eps)^{-1}$ with symbol $\dfrac{(\eps\ell)^2}{m(\eps\ell)}$ is well defined, maps from $H^2$ into $L^2$, and satisfies the estimate
\[
\|(\M^\eps)^{\-1}-d^{-1}I\|_{H^2 \to L^2} = \rmO(\eps^2).
\]
where $I$ denotes the identity operator that maps $H^2$ into $L^2$.
\end{lemma}
Our main result, Theorem \ref{MainRes}, will be proved by the following proposition, which solves the reduced bifurcation equation \eqref{1dnl} from the Lyapunov-Schmidt reduction in lemma \ref{Lemuh}.


\begin{proposition}\label{prop} Let $v_*$ be the unique even solution to the ordinary differential equation $dv^{''} - v +v^2 = 0$ which satisfies $v(0)>0$, $\lim_{x\to \pm \infty} v(x) = 0$. If $\eps_1>0$ is sufficiently small, then for $0<|\eps|<\eps_1$, there exist a family of solutions to \eqref{1dnl} of the form $v_c(\cdot;\eps) = v_*(\cdot)+w(\cdot; \eps)$. Here $w=w(\cdot,\eps) \in H^2_{e}$ is a family of correctors parametrised by $\eps $ such that $\|w(\cdot,\eps)\|_{H^2} \to 0$ as $\eps \to 0$.
\end{proposition}

%First, by assumptions on $f$, the Taylor expansion for $f$ near $(u,\mu)=(0,0)$ is
%\[
%f(u,\mu) = A u\mu - B u^2 + O(\mu^2 u, \mu u^2, u^3)
%\]
%where $A = f_{u\mu}(0,0), B=-f_{uu}(0,0)$, for definiteness we assume here that $A,B>0$, then we rescale $u$ and $\mu$ by $u \mapsto su$ and $\mu \mapsto t\mu$, where $s,t$ are constants to be chosen, then we have
%\[
%-su + sK\ast u = Ast u\mu -Bs^2u^2 + O(\mu^2 u, \mu u^2,u^3)
%\]
%cancel out $s$, we see choose $s=1/B$ and $A=1/t$ lead to the following equation for $u$ %and $\mu$:
%\[
%-u+K\ast u=f(u; \mu) = \mu u - u^2 + O(\mu u^2,\mu^2u,u^3),
%\] 
%then we rescale $u(x) = \mu v(\sqrt{\mu}x)$, we have an equation in $v$: 
%\begin{equation}  \label{scl nl}
%-v(y) + K_\eps \ast v (y) = \eps^2(v-v^2)+O(\eps^4v)
%where $\eps^2 = \mu$ and $K_\eps = \eps^{-1}K(\cdot/\eps)$, and $y =\eps x$. From this point we focus on $\eqref{scl nl}$.

%Dividing by $\eps^2$, and take Fourier transform of both sides of the equation (we assume we are solving in $H^2(\R)$.)

%\[
%\frac{-1+\hat{K}_\eps(\ell)}{-\eps^2\ell^2}(-\ell^2)\hat{v}(\ell) = \widehat{v-v^2}(\ell) %+ O(\eps^2 \hat{v})
%\] 

%We define the operator $M_\eps$ so that the Fourier multiplier $\widehat{M}_\eps(\ell) = \frac{-1+\hat{K}_\eps(\ell)}{-\eps^2\ell^2}= \frac{-1+\hat{K}(\eps\ell)}{-\eps^2\ell^2}$, by assumptions on $K$, we 
%know $M_\eps$ is bounded as an operator from $L^2$ to $L^2$, but $\sup |\widehat{M_\eps}(\ell)|$ does not necessarily go to zero as $\eps \to 0$.

%We take inverse Fourier transform and get back the equation in physical space:
%\begin{equation}\label{eq phy}
%M_\eps v''(y) = v(y)-v^2(y) + O(\eps^2 v)
%\end{equation}

\begin{Proof}
We substitute the ansatz $v_c = v_* + w$ into \eqref{1dnl}, where $v_*$ is as stated in the lemma and $w \in H^2$. We will determine an equation for $w$ and $\eps$ and show that it can be solved using Newton iteration scheme near $(w,\eps)=(0,0)$.
First, for the term $\eps^{-2}M^\eps v_c$ with $v_c \in H^2$, we apply Fourier transform to obtain
\[
\eps^{-2}m(\eps\ell)\widehat{v}_c(\ell) = -\frac{m(\eps\ell)}{(\eps\ell)^2}(-\ell^2)\widehat{v}_c(\ell) = \widehat{-\M^\eps v_c^{''}},
\]
and equation \eqref{1dnl} becomes
\[
0 = -\M^\eps v_c^{''} + \left(L_{cc}^\eps a_{101}^c+L_{ch}^\eps a_{101}^h\right)v_c+\left(L_{cc}^\eps a_{110}^c+L_{ch}^\eps a_{110}^h\right)v_c^2 + \Rm(v_c,\psi;\eps),
\]
where $\Rm(v_c,\psi;\eps)$ contains all the terms of order $\eps^2$ and higher,
\[
\Rm(v_c,\psi;\eps) =\sum_{j=c,h} L_{cj}^\eps\left[ a_{011}^j\psi+a_{200}^j v_c\psi+a_{020}^j [\psi,\psi]+\eps^{-4}\Rm_j(v_c,\psi;\eps)\right].
\]
Indeed, for $w$ with $\|w\|_{H^2}$ close to $0$ and $\eps$ small, $\Rm$ satisfies the estimate $\|\Rm\|_{H^2} = \rmO(\eps^2)$.  To see this, we apply Lemma \ref{Lemuh} with $r = 2\|v_*\|_{H^2}$, which is finite since $v_*(x)$ is exponentially decreasing to $0$ as $x\to \infty$.  We then obtain $\psi = \psi(v_*+w,\eps)$ which satisfies $\|\psi(v_*+w,\eps)\|_{H^2(\R;\R^{n-1})} = \rmO(\eps^2)$ as $\eps \to 0$.

The linear operators $L_{cc}^\eps, L_{ch}^\eps$ are uniformly bounded in $\eps$, so that we have
\[
\left\|\sum_{j=c,h} L_{cj}^\eps\left( a_{011}^j\psi+a_{200}^j v_c\psi+a_{020}^j [\psi,\psi]\right)\right\|_{H^2} \le C(\|\psi\|_{H^2(\R;\R^{n-1})}+\|\psi\|_{H^2(\R;\R^{n-1})}^2) = \rmO(\eps^2).
\]

On the other hand, the remainders $\Rm_c$ and $\Rm_h$ satisfy $\|\Rm_c\|_{H^2}= \rmO(\eps^6), \|\Rm_h\|_{H^2(\R;\R^{n-1})} = \rmO(\eps^6)$ uniformly for $v_*$ and $w$ such that $v_* +w \in B_r$ as $\eps \to 0$ by Lemma \ref{Lemuh}. Therefore we conclude that $\|\Rm(v_c,\psi;\eps)\|_{H^2} = \rmO(\eps^2)$ as $\eps \to 0$ for $v_c=v_*+w$.
 
Next, add the equation $dv_*^{''}-v_*+v_*^2 =0$ to the right hand side of \eqref{1dnl} and precondition with the operator $(\M^{\eps})^{-1}$. Set $\alpha^\eps = L_{cc}^\eps a_{101}^c+L_{ch}^\eps a_{101}^h, \beta^\eps=L_{cc}^\eps a_{110}^c+L_{ch}^\eps a_{110}^h$ and we find
\begin{align}
0 &=(\M^\eps)^{-1}\left[ (d-\M^\eps)v_*^{''} -\M^\eps w^{''}+\alpha^\eps(v_*+w)-v_*+\beta^\eps(v_*+w)^2+v_*^2 + \Rm \right] \nonumber\\
&=[(\M^{\eps})^{-1}-d^{-1}]\M^\eps dv_*^{''}-w^{''}+(\M^{\eps})^{-1}\left((\alpha^\eps-1)v_*+\alpha^\eps w)+(\beta^\eps+1)v_*^2+\beta^\eps(2v_*w+w^2) +\Rm\right), \label{splfy nl}
\end{align}


We denote the right hand side of \eqref{splfy nl} as $ F(w,\eps)$. Our goal is to set up a Newton iteration scheme to solve $ F(w,\eps) =0$ for $w$ in terms of $\eps$ as a fixed point problem.

Following the strategy of Lemma \ref{Lemuh}, we shall show
\begin{enumerate}
\item $\|F(0,\eps)\|_{L^2} \to 0$ as $\eps \to 0$.
\item $F(w,\eps)$ is continuously differentiable in $w$ and $D_wF(0,\eps): H^2_{e} \to L^{2}_{e}$ is uniformly invertible in $\eps$.
\end{enumerate}
For $(i)$, we compute
\[
F(0,\eps) = [(\M^{\eps})^{-1}-d^{-1}]\M^\eps dv_*^{''}+(\M^\eps)^{-1}[(\alpha^\eps-1)v_*+(\beta^\eps+1)v_*^2+\Rm(v_*,\psi;\eps)].
\]

By Lemma \ref{estmult}, $\|\left((\M^\eps)^{-1} - d^{-1}\right)v\|_{L^2} \le \eps^2\|v\|_{H^2}$ for any $v\in H^2$. Note that $v_*^{''}(x)$ is smooth and decays to $0$ exponentially as $x\to \infty$, while $\M^\eps : H^2 \to H^2$ is bounded since its symbol is a bounded function. We find $\M^\eps dv_*^{''} \in H^2$ and, consequently $\|(\M^{\eps})^{-1}-d^{-1}]\M^\eps dv_*^{''}\|_{L^2} = \rmO(\eps^2)$.

Moreover, by our rescaling, $\alpha^\eps v \to \alpha v= v$ and $\beta^\eps v\to \beta v=-v$ for any $v \in H^2$, and the remainder $\Rm=\Rm(v_*,\psi;\eps)$ satisfies $\|\Rm\|_{H^2} = \rmO(\eps^2)$ as proved earlier. Hence, we have 
\begin{align*}
&\|(\M^\eps)^{-1}[(\alpha^\eps-1)v_*+(\beta^\eps+1)v_*^2+\Rm(v_*,\psi;\eps)]\|_{L^2} \le \\
\le & \|d^{-1}(\alpha^\eps-1)v_*+d^{-1}(\beta^*+1)v_*^2\|_{L^2} + \|d^{-1} \Rm\|_{H^2} + \rmO(\eps^2).
\end{align*}
Therefore we conclude that $\|F(0,\eps)\|_{L^2} \to 0$ as $\eps \to 0$.

For $(ii)$, we first verify that $F$ is continuously differentiable in $w$ from $H^2$ to $L^2$. Indeed, fix $w_0\in H^2$, for $h \in H^2$ we observe that $D_wF(w_0,\eps)h:H^2 \to L^2$ is given by
\[
D_wF(w_0,\eps)h = -h^{''}+(\M^\eps)^{-1}\left[(a^\eps h)+2v_*\beta^\eps h + 2w_0h)+D_w\Rm h\right].
\]
Recall that the nonlinear function $\Nl$ is smooth in its arguments, so  that$D_wF(w_0,\eps)$ is actually smooth in the variable $w_0$.

Now, at $w_0 = 0$, we see that, $D_wF(0,\eps)h \to -h^{''}+d^{-1}[h-2v_*h]$ in $L^2$ as $\eps \to 0$ for $h \in H^2$, because $\|D_w\Rm h\|_{H^2} = \rmO(\eps^2)$ as remarked earlier. By lemma \ref{Nondegen} in the Appendix, the operator $\cL : H^2_e \to L^2_e$, given by 
\[
\cL h = -h^{''}+d^{-1}[h-2v_*h],
\] is bounded invertible. We notice that $D_wF(0,\eps)$ is a small perturbation of $\cL$, therefore invertible with uniform bounds on the inverse for $\eps$ small enough.

We now set up the Newton iteration scheme, define $\tilde{\cS}: H^2 \to L^2$ through
\[\tilde{\cS}(w,\eps) = w-D_wF(0,\eps)^{-1}[F(w,\eps)].\]
As in \ref{Lemuh}, we obtain $w=w(\eps)$ which solves $F(w(\eps),\eps) = 0$ for $\eps $ small enough and satisfies $\|w(\eps)\|_{H^2} \to 0$ as $\eps \to 0$.
\end{Proof}

Finally, we prove Theorem \ref{MainRes}.
\begin{Proof}[ of Theorem 2.1.] We now write the tildes for the rescaled variables. From proposition \ref{prop}, we know that \eqref{1dnl} has a solution of the form $\tilde{v}_c(\cdot) = v_*(\cdot)+w(\cdot;\eps)$. Together with $\tilde{v}_h = \psi(\tilde{v}_c,\eps)$, reverting the rescaling, we obtain $v_c(\cdot) = -\frac{\alpha}{\beta}\mu \tilde{v}_c(\sqrt{\alpha\mu }\cdot)$ and $v_h(\cdot) = \alpha\mu \tilde{v}_h(\sqrt{\alpha\mu}\cdot)$ as solutions to \eqref{exeqnu0} and \eqref{exeqnuh}.

 Now, recall that $V=(v_c,v_h)^T$ and the original variable $U$ are related by $U= QV$ where $Q$ is defined in the proof of Lemma \ref{Lem1}. We conclude that $U(\cdot)=v_c(\cdot)\mathcal{E}_0+v_{\perp}(\cdot)$, where $v_{\perp}$ takes values in the complement of $\mathcal{E}_0$. The behavior of $v_c,v_{\perp}$ as $\mu \to 0$ is a direct consequence of Lemma \ref{Lemuh} and Proposition \ref{prop}.
\end{Proof}
\section{Discussion}
In this section we make remarks on our results, and discuss possible continuations.
\paragraph{Exponential localization.}
The convolution kernel $\K$ in our assumption is in general not exponentially localized. Therefore, the solution we obtained will not be exponentially localized in general. Consider the scalar equation $-u+K\ast u=\mu u -u^2$ to which our assumptions apply. If $u$ is an exponentially localized solution, then $u+\mu u-u^2 = K\ast u$ will also be exponentially localized, however, this is not possible since $K$ has merely algebraic decay.

\paragraph{Extension to higher spatial dimensions.}
As is clear from the proof, we did not use $ x\in \R$. We could then consider in general the case $x\in \R^k$, assuming a radially symmetric kernel. Using the model equation $\Delta u = \mu u -u^p$, where existence, uniqueness and nondegenracy of a radial ground state is well known for $1<p<\frac{k+2}{k-2}$ (e.g. see the appendix in \cite{wei2013mathematical}), we should be able to show similar results hold. 


\paragraph{Stability of the bifurcating branches.}
If we consider evolution equation of the form \[U_t = U+\K\ast U-\Nl(U;\mu), \hspace{0.1in}\text{ or }\hspace{0.1in} U_{tt} =U+\K\ast U-\Nl(U;\mu),\] then a natural continuation is to study the stability properties of the solutions bifurcated from the trivial state. The generic condition \eqref{muvCoe} and \eqref{QuadCoe} resembles the condition for a transcritical bifurcation in the equation $u^{''}=\mu u-u^2$, although we only assumed $\mu>0$ in this paper, a similar argument can be used to show nontrivial solutions bifurcate from $U=0$ as $\mu$ becomes negative, 
\section*{Appendix}
In this appendix we prove the following result.
\begin{lemma}\label{Nondegen}
The operator $\cL : H^2_e \to L^2_e$ defined by  $\cL h =  h''-d^{-1}[h-2v_*h]$ is bounded invertible.
\end{lemma}
\begin{Proof}
The kernel of $\cL$ consists of bounded solutions of the differential equation $dh^{''}-h+2v_* h = 0$. This equation is satisfied by the function $v'_*(x)$, we claim that it is the unique element which spans the kernel of $\cL$ in $H^2$, to see this, we rewrite the equation as a nonautonomous linear first order system 

\[
\dot{Y}(x) = A(x)Y(x), \text{ wtih }Y(x) = \begin{pmatrix}
h(x)\\
h'(x)
\end{pmatrix} \text{ and } A(x) = \begin{pmatrix}
0&1\\
d^{-1}(1-2u_*(x))&0
\end{pmatrix}.
\]

Now, as $A(x)$ converges to the hyperbolic matrix $A_{\infty}=\begin{pmatrix}
0&1\\
d^{-1}&0
\end{pmatrix}$ as $x \to \pm \infty$, by the robustness of exponential dichotomy, we see $A(x)$ possess an exponential dichotomy on $\R^+$ and $\R-$, let $E_+^s$ denote the image of the stable projection of the exponential dichotomy on $\R^+$ and $E_-^u$ denote the unstable projection of the exponential dichotomy on $\R^-$. The kernel of $L$ is isomorphic to $E_+^s \cap E_-^u$, but $E_-^u$ and $E_+^s$ are one-dimensional since the stable and unstable eigenspace of $A_{\infty}$ are both $1$, hence the kernel is at most one-dimensional, since $u_*'$ already lies in the kernel, we see the dimension of the kernel is exactly one.

Next we show that $\cL$ is Fredholm with index zero from $H^2$ to $L^{2}$ by writing it as the compact perturbation of an invertible operator. For $h \in H^2$, define operators $\cL_1$ and $\cL_2$ by
\[
\cL_1h = (dh''-h), \hspace{0.2in} \cL_2 h = (2v_*)h.
\] 

We first show $\cL_2: H^2 \to L^2$ is compact. Fix an integer $k>0$, let $\chi_k$ be a positive smooth cutoff function which equals $1$ on the interval $[-k,k]$ and vanishes outside of $[-2k,2k]$. Consider the sequence of operators $\cL_2^{(k)} : H^2 \to L^2$ defined by $\cL_2^{(k)} h = \chi_k (2v_*)h$. We claim that $\cL_2^{(k)}$ are compact operators for each $k$ and converges to $\cL_2$ in the operator norm.

To see $\cL_2^{(k)}$ are compact for each $k$, let $\{h_i\}_{i=1}^{\infty}$ be a bounded sequence of functions in $H^2$ with $\|h_i\|_{H^2} \le 1$. We want to show that there exist a subsequence $h_{i'}$ so that $\cL_2^{(k)} h_{i'} = \chi_k (2v_*)h_{i'}$ is convergent in $L^2$. From the Sobolev embedding $H^2 \to \mathscr{C}^1$, the $h_i$ and their derivatives are uniformly bounded on $\R$. Therefore, $\cL_2^{(k)}h_i = \chi_k(2v_*)h_i$ is a sequence of functions that is uniformly bounded and equicontinuous on the compact interval $[-2k,2k]$, we conclude from the Arzela-Ascoli theorem that there exist a subsequence $i'$ such that $\cL_2^{(k)}h_{i'}$ converges uniformly on $[-2k,2k]$. Since $\cL_2^{(k)}h_{i'}$ are continuous and have a common compact support $[-2k,2k]$, it is an convergent sequence in $L^2$ as well. Therefore $\cL_2^{(k)}$ is compact for each $k$.

To see $\cL_2^{(k)} \to \cL_2$ in the operator norm, fix $h \in H^2$ with $\|h\|_{H^2} =1$, we have
\[
\|(\cL_2^{(k)}-\cL_2)h\|_{L^2} = \|(\chi_k-1)(2v_*)h\|_{L^2}\le  \sup_{x\in \R}(\chi_k(x)-1)(2v_*(x)) \|h\|_{L^2}.
\]

Since $\chi_k = 1$ on $[-k,k]$ and $v_*(x) \to 0$ as $x\to \pm \infty$, it follows that $\displaystyle \sup_{x\in \R}(\chi_k(x)-1)(2v_*(x)) \to 0$ as $k \to \infty$. Thus $\|\cL_2^{(k)} - \cL_2\|_{H^2 \to L^2} \to 0$ and we conclude that $\cL_2:H^2 \to L^2$ is compact.


Next we show that the second order differential operator $\cL_1 h = dh'' - h$ is invertible as an operator from the space $H^2_{e} \to L^{2}_{e}$. Fix $f \in L^2$, using Fourier transform, we obtain
\[
-(1+d\ell^2)\widehat{h}(\ell) = \widehat{f}(\ell).
\]
As a consequence $h = \mathcal{F}^{-1} \{-(1+d\ell^2)^{-1}\widehat{f}\}$, the inverse for $\cL_2$ is given by the operator with symbol $(1+d\ell^2)^{-1}$, so it is an bounded operator from $L^2$ to $H^2$.

Therefore, we conclude that $\cL$ is Fredholm with index zero from $H^2$ to $L^{2}$, since the kernel is spanned by the odd function $v_*'$, it is invertible from $H^2_{e}$ to $L^{2}_{e}$.
\end{Proof}
\bibliographystyle{plain}
\bibliography{nlBfCs}


\end{document}
