\documentclass[letterpaper,11pt]{article}

\usepackage{ucs}
\usepackage[utf8x]{inputenc}
\usepackage{graphicx}
\usepackage{amsfonts}
\usepackage{dsfont}
\usepackage{amssymb}
\usepackage{amsmath,mathrsfs}
\usepackage{amsthm}
\usepackage{enumerate}
\usepackage{stmaryrd}
\usepackage{fullpage}
\usepackage{ifthen}
\usepackage{subfigure}
\usepackage{epic}
\usepackage{authblk}
\usepackage{textcomp}
\usepackage[small]{caption}
%\usepackage{mathtools}


\usepackage[hypertexnames=false,colorlinks=true,linkcolor=blue,citecolor=blue]{hyperref}
\usepackage[numbers,comma,square,sort&compress]{natbib}
\usepackage[letterpaper,text={7in,9in},centering]{geometry}


\usepackage{color}
\usepackage{titlesec}
\setlength{\parindent}{0.0in}
\setlength{\parskip}{1.0ex plus0.2ex minus0.2ex}
\renewcommand{\baselinestretch}{1.1}
\graphicspath{{eps/}{pdf/}}
%\setcaptionmargin{0.25in}
\def\captionfont{\itshape\small}
\def\captionlabelfont{\upshape\small}

\renewcommand{\labelenumi}{(\roman{enumi})}

\newcommand{\bqq}{\begin{equation}}
\newcommand{\eqq}{\end{equation}}
\newcommand{\bqs}{\begin{equation*}}
\newcommand{\eqs}{\end{equation*}}

\newcommand{\C}{\mathbb{C}}
\newcommand{\D}{\mathbb{D}}
\newcommand{\N}{\mathbb{N}}
\newcommand{\R}{\mathbb{R}} 
\newcommand{\Z}{\mathbb{Z}}

\newcommand{\rme}{\mathrm{e}}
\newcommand{\rmi}{\mathrm{i}}
\newcommand{\rmd}{\mathrm{d}}
\newcommand{\rmo}{{\scriptstyle\mathcal{O}}}
\newcommand{\rmO}{\mathcal{O}}
\newcommand{\eps}{\varepsilon}
\newcommand{\B}{\mathcal{B}}
\newcommand{\Rm}{\mathcal{R}}
\newcommand{\Nl}{\mathcal{N}}
\newcommand{\K}{\mathcal{K}}
\newcommand{\G}{\mathcal{G}}
\newcommand{\F}{\mathcal{F}}
\newcommand{\M}{\mathcal{M}}
\newcommand{\cS}{\mathcal{S}}
\newcommand{\cL}{\mathcal{L}}
\newcommand{\cJ}{\mathcal{J}}
\newcommand{\That}{\widehat{\mathcal{T}}}

\newcommand{\diag}{\operatorname{diag}}
\newcommand{\spa}{\operatorname{span}}


\numberwithin{equation}{section}

\newenvironment{Hypothesis}[1]%
  {\begin{trivlist}\item[]{\bf Hypothesis #1 }\em}{\end{trivlist}}

\renewcommand{\arraystretch}{1.25}


% Define Theorem Styles ----------------------------------
\theoremstyle{plain}
\newtheorem{theorem}{Theorem}[section]
\newtheorem{proposition}[theorem]{Proposition}
\newtheorem{lemma}[theorem]{Lemma}
\newtheorem{corollary}[theorem]{Corollary}
\newtheorem{conjecture}[theorem]{Conjecture}
\newtheorem{main}[theorem]{Main Result}
\newtheorem{rmk}[theorem]{rmk}
\theoremstyle{remark}
\newtheorem*{remark}{Remark}

\newcommand{\etal}{\textit{et al.}\ }

\newcommand{\greg}[1]{%
  {\color{blue}\textbf{Greg:} #1}%
 }
 
\newcommand{\arnd}[1]{%
  {\color{red}\textbf{Arnd:} #1}%
 }

\newenvironment{Proof}[1][.]%
 {\begin{trivlist}\item[]\textbf{Proof#1 }}%
 {\hspace*{\fill}$\rule{0.3\baselineskip}{0.35\baselineskip}$\end{trivlist}}

\renewcommand\labelitemi{$\bullet$}


\title{Bifurcation of coherent structures in nonlocally coupled system}
\author{Arnd Scheel and Tianyu Tao}
\date{2017}
\begin{document}
\maketitle
\begin{abstract}

Motivated by models for neural fields, we study the existence of pulses  bifurcating from a spatially homogeneous state in nonlocally coupled systems of equations. More specifically, we look at equations of the form $U + \K\ast U = \Nl(U;\mu)$, where $\Nl$ encodes nonlinear terms, $\K$ is an even matrix convolution kernel. Assuming the presence of neutral modes, that is, solutions of the form $u\sim \exp(i \ell x)$ to the linear part, we show under appropriate assumptions on the nonlinearity and the unfolding in $\mu$ that pulses bifurcate. Such an analysis is carried out using center manifold reduction, when coupling is local, say, $\K=\delta''$. Here, we rely on functional analytic methods using predictors from formal expansions and correctors obtained after preconditioning the nonlinear system.
\end{abstract}

\section{Introduction}
In this paper we study the equation
\begin{equation} \label{system}
U+\K\ast U = \Nl(U;\mu) ,
\end{equation}
where $U=U(x):\R^n \to \R^k$, and $\K\ast U$ stands for matrix convolution,
\[
(\K\ast U(x))_i = \sum_{j=1}^m \int_{\R^n} \K_{i,j}(x-y)U_j(y)dy, \hspace{0.2in} 1\le i\le k,
\]

and $\Nl(U;\mu)$ encodes nonlinear terms which depend on a parameter $\mu\ge 0$. We assume $\Nl(0;\mu) =0$, so $U\equiv 0$ is a trivial solution for all $\mu$.

We are interested in possible solutions that bifurcate from the trivial solution $U\equiv 0$ as one increases the parameter $\mu$. The existence of such solutions bears important physical meaning and poses interesting mathematical questions.

One of the examples arises when studying stationary or traveling-wave solutions to neural field equations which are typically modeled by equations of the form
\begin{equation}\label{NFE}
\frac{du}{dt} =- u+\K\ast S(u),
\end{equation}
where $u=u(t,x) \in \R$ represents the local activity of a population of neurons at position $x$ in the cortex, and $S$ is a firing rate function, the kernel $\K$ encodes the connectivity. Stationary solution to \eqref{NFE} are thought to be associated to short term memory, providing motivation for extensive studies of such solutions.

Another area where nonlocal equations such as \eqref{system} appear is when studying phase transition. Examples include equation of the form
\begin{equation}\label{PT}
u_t = -u+J\ast u - f(u), 
\end{equation}
where now $u = u(x,t) \in \R$ is an order parameter describing the state of a solid material at position $x$ and time $t$, $J$ is a convolution kernel with $\int_\R J=1$ and $f$ is a bistable nonlinearity with zeros $-1,1$ and $\alpha \in (-1,1)$. As a consequence, $u = \pm 1$ are steady state solution to \eqref{PT}, they may represent two orientations of a perfect crystal. Any values between $u=\pm 1$ can represent intermediate states of the crystal. For example, a traveling wave solution $u(x)$ connecting $u=-1$ and $u=+1$ represents the process of ``invasion'' of one pure state into the other. The advantage of using the nonlocal dispersal term $-u+J\ast u$ instead of the diffusion term $u_{xx}$ is that more general types of interactions between states at nearby locations in the medium can be accounted for.

In terms of techniques, one popular approach is to use a special form of the kernel $\K$ that has ``rational'' Fourier transforms: and reduces the nonlocal equation to a system of ordinary differential equation (ODE), where classical dynamical system methods are available.

As a specific example, spatially localized solutions (referred to as ``bumps'' or ``homoclinics'') in equations of the form 
\begin{equation}\label{Laing}
\partial_t a(x,t) = -a(x,t)+\int_{-\infty}^{\infty} w(x-y)S(\mu a(y,t))dy,
\end{equation}
were investigated. Laing et al. \cite{laing2003pde} considered the case $S(\cdot) = 2\exp(-r/(\cdot-\theta)^2)H(\cdot-\theta)$, with $H$ the Heaviside function and $w(x) = \exp(-b|x|)(b\sin |x|+\cos x)$. 
Faye et al. \citep{faye2013localized} studied the case $S(\cdot) = (1+e^{-\cdot+\theta})^{-1}$ and general $w$ such that $\widehat{w}(\xi) = R(\xi^2)/Q(\xi^2)$, where $R,Q$ are polynomials in $\xi^2$ satisfying $\deg R<\deg Q$. Using this approach, they are able to transform the nerual field intego-differential equation into a partial differential equation (PDE). Restricting to stationary solutions, the PDE reduces further into a system of ODE. Laing et al. \citep{laing2003pde} proceed with numerical investigations, whereas Faye et al. \citep{faye2013localized} used normal form theory and center manifold reduction (see for example \cite{haragus2010local}) to find the desired solution.

Another example is in \cite{faye2013existence}, where Faye considered a variant of the FitzHugh–Nagumo equation of the form
\begin{subequations}
\begin{eqnarray}
\tau u_t(x,t) &=& -u(x,t)+\int_\R J(x-y)q(y,t)S(u(y,t))dy, \\
\eps^{-1} q_t(x,t) &=& 1-q(x,t)-\beta q(x,t) S(u(x,t)) ,
\end{eqnarray}
\end{subequations}
with $S(u) = (1+e^{-\lambda(u-\kappa})^{-1}, J(x) = b/2\exp(-b|x|)$, and parameters $\lambda,\kappa,b$. Using the property $\widehat{J}(\ell) = b^2/(b^2+\ell^2)$ and looking for traveling wave solutions, Faye was able to reduce the neural field equation into a system of ODEs with a singular perturbation structure. He then used geometric singular perturbation theory (GSPT) to find the desired solution.

Other important techniques for demonstrating the existence of travelling waves, not necessarily spatially localized (refered to as ``fronts'' and ``backs''), include homotopy arguments. In \cite{Bates1997}, Bates et al. established a homotopy between equation \eqref{PT} and the reaction-diffusion equation $u_t=u_{xx}+f(u)$, where the existence of travelling wave is well-known since the pioneering work of \cite{KPP}. Using comparison principles, in \cite{chen1997existence}, Chen showed the existence of traveling fronts that connect $0$ and $1$ for a very general class of nonlocal evolution equations of the form $u_t(x,t) = \mathcal{A}[u(\cdot,t)](x)$, with $\mathcal{A}$ a nonlinear operator that satisfies various assumptions, among which the most important is the comparison principle: if $u_t \ge \mathcal{A}[u],v_t \le \mathcal{A}[v]$ and $u(\cdot,0) \ge v(\cdot,0)$ but not equal to each other, then $u(\cdot,t)>v(\cdot,t)$ for all $t>0$. Chen constructed the solution by first choosing an appropriate initial data, then evolve it according to the equation, and show that the function of the form $u(\cdot+\xi(t),t)$ will converge to the profile of a traveling wave as $t\to \infty$, where $\xi(t)$ is chosen so that $u(\xi(t),t)=1/2$.


Motivated by \cite{pulseNLFHN}, where the authors studied a nonlocal system of the form 
\begin{subequations}
\begin{eqnarray}
u_t(x,t) &=& -u(x,t) + \int_{\R} \K(x-y)u(y,t)dy+f(u(x,t))-v(x,t),\\
v_t(x,t) &=& \eps(u(x,t)-\gamma v(x,t)),
\end{eqnarray}
\end{subequations}
and proved the existence of travelling front solutions using an extension of the singular-perturbation method. They constructed the slow manifold and the singular solution using cut-off functions and implicit-function theorem, replacing the more common geometric dynamical system methods. In the present paper we continue this functional-analytic approach, and focus on the simpler system \eqref{system}. We will use Fourier transform to rewrite equation \eqref{system} in a form that relates it to the ODE $u''=-\mu u+u^2$, and set up a Newton iteration scheme to continue the solution for $\mu>0$ sufficiently small.

\paragraph{Outline.}The remainder of the paper is organized as follows: after introducing the notations we will use in this paper, in section $2$ we prove our main results, in section $3$ we discuss possible generalizations of our result and future directions. The appendix contains an elementary proof of the nondegeneracy of the operator $\partial_{xx}-d^{-1}(1-2u_*)$. 

\paragraph{Notation.}

For a vector $u=(u_1,\ldots,u_k) \in \R^k$, we write $|u|$ to denote its usual Euclidean norm $|u| = \sum_{i=1}^{k} u_i^2$. We also use the standard multi-index notation in $\R^n$, that is we have $\alpha = (\alpha_1,\cdots,\alpha_n)$ with $\alpha_i \in \{0,1,\ldots \}$ and $\alpha! = \alpha_1!\cdots\alpha_n!$, $|\alpha|=\alpha_1+\cdots+\alpha_n$. So that 
$D^\alpha u = \dfrac{\partial^{|\alpha |} u}{\partial_{x_1}^{\alpha_1}\cdots \partial_{x_n}^{\alpha_n}}$.

We shall use the standard Sobolev spaces on $\R^n$ with values in $\R^k$, which are denoted by $W^{\ell,p}(\R^n; \R^k)$ or simply $W^{\ell,p}(\R^n)$ when $k=1$ or even $W^{\ell,p}$ whenever it is convenient to do so and does not cause confusions. For $\ell \ge 0$ and $1\le p \le \infty$
\[
W^{\ell,p}(\R^n;\R^k) := \{ u \in L^p(\R^n;\R^k): \partial^\alpha u \in L^p(\R^n;\R^k), 1\le |\alpha| \le \ell \},
\]
with norm
\[
\|u\|_{W^{\ell,p}(\R^n;\R^k)}=
\begin{cases}
\left(\sum_{1\le |\alpha| \le \ell} \|\partial^\alpha u \|_{L^p(\R^n;\R^k)}\right)^{1/p}, \hspace{0.1in}1\le p<\infty \\
\max_{1\le |\alpha| \le \ell}\|\partial^\alpha u\|_{L^\infty(\R^n;\R^k)}, \hspace{0.4in} p=\infty.
\end{cases}
\]
We use $H^\ell(\R^n;\R^k)$ to denote the space $W^{\ell,2}(\R^n;\R^k)$, and $W^{\ell,p}_r(\R^n;\R^k)$ the subspace of $W^{\ell,p}(\R^n;\R^k)$ which consists of radially symmetric functions in $W^{\ell,p}(\R^n;\R^k)$. We will also use $\mathscr{C}_b^\ell(\R^n;\R^k)$ to denote the space of $\ell-$times bounded continuously differentiable functions for $\ell=0,1,\ldots,\infty$.
 
Finally, we use the usual Fourier transform on $\R^n$, 
\[
\widehat{f} (\xi)= \int_{\R^n} f(x)e^{-2\pi i x\cdot \xi}dx
\]
 for a Schwartz function $f$, which extends by isometry to all $f \in L^2(\R^n;\R^k)$. 




\section{Hypothesis and the main result}

We are interested in nontrivial solutions bifurcating from the homogeneous state $U = 0$ as one increases the parameter $\mu$. Before stating the main result, we introduce our hypothesis.

\paragraph{Assumptions on the linear part.}We state our first main hypothesis.
\begin{Hypothesis}{(H1).}Let $I_k$ denote the identity matrix of size $k$. We make the following assumptions on $\K$:
\item (i)  The entries of $\K$ are radially symmetric, and has finite second moments, that is, for $1\le i,j\le k$ and all $|\alpha|\le 2$ we have
\[
 \K_{i,j}(x) \in L^1_r(\R^n), \hspace{0.3in}\text{and}\hspace{0.3in}  x^\alpha \K_{i,j}(x) \in L^1(\R^n).
\]


\item [(ii)] For $\xi \in \R^n$, write $s$ for the radial variable $|\xi|$, we define the characteristic equation $\mathcal{D}(s)$ as
\[\mathcal{D}(s) = 
\mathcal{D}(|\xi|):=\det(I_k+\widehat{\K}(|\xi|))= \det(I_k+\widehat{\K}(\xi)),\hspace{0.1in} \textit{ for } \xi \in \R^n.\] We then make the following assumptions on the characteristic equation.
\begin{itemize}
\item $\mathcal{D}(0)= 0$ with $\mathcal{D}''(0) \neq 0$;
\item $\mathcal{D}(\xi) \neq 0$ for all $\xi \neq 0$.
\end{itemize}
\end{Hypothesis}


By $(i)$ and properties of the Fourier transform, entries of $\widehat{\K}$ are radially symmetric, $\mathscr{C}^2$ functions in $\xi$, which justifies the validity of taking the second derivative of $\mathcal{D}$ in $(ii)$. Since $\widehat{\K}$ is radial, we shall write $\widehat{\K}(s)$ for $\widehat{\K}(\xi)$. From $(ii)$ we conclude that $I_k+\widehat{\K}(\xi)$ is invertible for all $\xi$ except $0$. At $\xi = 0$, it readily follows from the above that the kernel of $I_k + \widehat{\K}(0)$ is one dimensional. Therefore there exist $\mathcal{E}_1\in \R^k$ and $\mathcal{E}_1^* \in \R^k$ such that
\[
[I_k+\widehat{\K}(0)]\mathcal{E}_1 = \mathcal{E}_1+\widehat{\K}(0) \mathcal{E}_1 = 0,  \hspace{0.1in} [I_k+\widehat{\K}(0)]^T\mathcal{E}_1^* = \mathcal{E}_1^*+\widehat{\K}(0)^T \mathcal{E}_1^* =0, \hspace{0.1in} \langle \mathcal{E}_1, \mathcal{E}_1 \rangle = \langle \mathcal{E}_1^*, \mathcal{E}_1^*\rangle=1,
\]
where $A^T$ denotes the transpose of the matrix $A$ , and $\langle \cdot,\cdot \rangle$ denotes the standard inner product on $\R^n$ given by
\[
\langle u,v\rangle = \sum_{i=1}^{n} u_iv_i, \text{ for any }u=(u_i)_{i=1}^n\in \R^n, \text{ and }v=(v_i)_{i=1}^n \in \R^n.
\]

\paragraph{Assumptions on the nonlinear part.}

We state our second main hypothesis.
\begin{Hypothesis} {(H2).} We assume $\Nl=\Nl(U;\mu):\R^k \times \R^+ \to \R^k $ is of class  $\mathscr{C}^{\infty}(\R^k \times \R^+; \R^k)$. Moreover, we require
\begin{enumerate}
\item $\Nl(0;\mu) = 0$ for all $\mu$.

\item The derivative of $\Nl$ satisfies the generic condition 
%\left(S_{cc}^\infty+ \widehat{J}_{cc}(0)\right)\frac{\partial^2}{\partial \mu \partial v_c}\tilde{\Nl}_c(0,0;0)+\left(S_{ch}^\infty+ \widehat{J}_{ch}(0)\right)\frac{\partial^2}{\partial \mu \partial v_c}\tilde{\Nl}_h(0,0;0) \neq 0,
\begin{eqnarray}
\alpha := \langle D_{\mu,U} \Nl(0;0)\mathcal{E}_1, \mathcal{E}_1^*\rangle \neq 0
 \label{muvCoe}, \\ 
\beta := \langle D_{U,U} \Nl(0;0)[\mathcal{E}_1,\mathcal{E}_1], \mathcal{E}_1^*\rangle \neq 0  \label{QuadCoe}.
\end{eqnarray}

%\left(S_{cc}^\infty+ \widehat{J}_{cc}(0)\right)\frac{\partial^2}{\partial v_c^2}\tilde{\Nl}_c(0,0;0)+\left(S_{ch}^\infty+ \widehat{J}_{ch}(0)\right)\frac{\partial^2}{\partial v_c^2}\tilde{\Nl}_h(0,0;0) \neq 0,
%where $S(0)$ is the matrix $S(\ell)$ constructed in lemma \ref{Lem1} evaluated at $\ell = 0$.

\end{enumerate}

\end{Hypothesis}

Note that by the above hypothesis, using Sobolev embedding, the superposition operator $U(\cdot) \mapsto \Nl(U(\cdot);\mu)$ maps $H^\ell(\R^n;\R^k)$ into $H^\ell(\R^n;\R^k)$ and is smooth for $\ell > n/2$. We include a proof of this fact in the appendix. Part $(ii)$ of our hypothesis is analogous to the generic transcritical bifurcation assumption (see chapter 6.6 of \cite{chow1982methods}), a typical example is $\Nl(U;\mu)=\mu U-U^2$ when $n=1$.


We can now state our main result.
 
\begin{theorem}\label{MainRes}Fix $n<6$ and $\ell>n/2+1$, assume hypothesis (H1) and (H2). If $\frac{1}{2}\langle \widehat{\K}^{''}(0)\mathcal{E}_1,\mathcal{E}_1^*\rangle := d>0$ and $\alpha>0$, there then exist a positive constant $\mu_0$ sufficiently small, such that \eqref{system} has a family of nontrivial solutions $U_*=U_*(\cdot;\mu) \in H_r^\ell(\R^n;\R^k)$ parametrized by $\mu\in (0,\mu_0) $, with the expression \begin{equation}\label{expan}
U_*(x;\mu) =  v_c(x;\mu)\mathcal{E}_1 + v_{\perp}(x;\mu)
\end{equation}
Here, $v_{\perp}$ takes value in the complement of $\mathcal{E}_1$, and satisfies $\|v_{\perp}\|_{H^\ell} = \rmO(\mu^2)$ as $\mu \to 0$. $v_c$ is a scalar function of the form 
$v_c(x; \mu)=-\frac{1}{\beta} (\alpha\mu)\left[ v_*(\sqrt{\alpha\mu}x)+w(\sqrt{\alpha\mu}x;\sqrt{\alpha\mu}) \right]$. Here $w(\cdot;\sqrt{\alpha\mu}) \in H^{\ell}_r$ is a corrector function with $\|w(\cdot;\sqrt{\alpha\mu})\|_{H^\ell} \to 0$ as $\mu \to 0$, and $v_*(\cdot)$ is the unique ground state solution to the equation $d\Delta v -u+u^2 =0$, with the condition 
$v_*(0)>0$ and $\lim_{|x|\to \infty}v_*(x)=0$.

\end{theorem}

First, let $T_0$ be the coordinate transformation which normalizes the second moment matrix $S_{ij}$, so that in the coordinate $y = T_0x$, the kernel $\tilde{\K}(y) = |\det T_0|\K(T_0y)$ satisfies
\[
\int x_ix_j\langle \mathcal{E}_1^*, \tilde{\K}(x)\mathcal{E}\rangle dx = 2\delta_{ij}
\]

note that the first moment of $\tilde{\K}$ vanishes, $\int y\tilde{\K}(y) = |\det T_0|T_0^{-1}\int  x\K(x) dx = 0$ if $x = T_0y$. 
\subsection{Normal form transformation}
We transform the operator $\widehat{\mathcal{T}}$ to a ``normal form'' via the following lemma.


\begin{lemma}\label{Lem1} There exist invertible $k \times k$ matrices $P, Q$, and a multiplier operator $L$ whose symbol $\widehat{L}(\xi) \in L^\infty$ such that
\[
\widehat{L}(\xi)P[I_n+\widehat{\K}(\xi)]Q = \diag\{m(\xi),I_{n-1} \},
\]
where $m(\xi) = \dfrac{|\xi|^2}{1+|\xi|^2}$.

\end{lemma}
\begin{Proof}
(for this argument assume $\K = \tilde{\K}$) should find a better notation for $\tilde{\K}...$

Set $\mathcal{T} := I + \K \ast$ so $\widehat{\mathcal{T}}(\xi) = I_k+\widehat{\K}(\xi)$, we divide our construction in 2 steps.

\paragraph{Step 1.}
 When $\xi = 0$, the rank of $\widehat{\mathcal{T}}(0)=I_k+\widehat{\K}(0)$ is equal to $k-1$ since it has a one dimensional kernel spanned by $\mathcal{E}_0$, it is standard (see for example \cite{roman2007advanced}) that there exist invertible matrices $P$ and $Q$ such that
\[
P\That(0)Q = \diag\{0,I_{k-1}\}.
\]

%In order to do computations later, we include here a construction of the matrices $P$ and $Q$. Let $\mathcal{E}_2,\ldots,\mathcal{E}_{k}$ be a set of vectors which span the complement of $\ker \mathcal{T}(0)$. Set $\mathcal{F}_i = \mathcal{T}(0)\mathcal{E}_i$ for $i=2,\ldots,k$. Define $Q$ to be the matrix whose column vectors are $\mathcal{E}_1,\ldots,\mathcal{E}_{k}$ (with respect to the standard basis in $\R^k$), and $P$ so that the columns of $P^{-1}$ are given by $\mathcal{E}_1^*,\mathcal{F}_2,\ldots,\mathcal{F}_{k}$. By the Fredholm alternative, we know that $\mathcal{E}_1^*$ is orthogonal to the subspace spanned by $\mathcal{F}_i$, that is $\langle\mathcal{E}_1^*,\mathcal{F}_i\rangle = 0$ for $i=2,\ldots,k$. 

% Then we have 
%\[
%P\mathcal{T}(0)Q = P\mathcal{T}(0)[\mathcal{E}_1,\ldots,\mathcal{E}_{k}] = P[0,\mathcal{F}_2,\ldots,\mathcal{F}_{k}]=\diag\{0,I_{k-1}\},
%\]
%which is the desired diagonalization.

By the assumption on finite second moment, $\widehat{\K}(\xi)_{i,j}$ are $\mathscr{C}^2$ functions. Taylor expand near $\xi = 0$, we find
\[
P\That(\xi)Q = \begin{pmatrix}
A(\xi)& B(\xi)\\
C(\xi)& D(\xi) 
\end{pmatrix},
\]
where $A(\xi) = |\xi|^2+o(|\xi|^2)$ exactly because the normalization condition $\int x_ix_j\langle \mathcal{E}_1^*, \tilde{\K}(x)\mathcal{E}\rangle dx = 2\delta_{ij}$, $B(\xi), C(\xi)$ are $k-1$ sized row, column vectors respectively, both satisfies $B(\xi),C(\xi) =\rmO(|\xi|^2)$, and $D(\xi) =I_{k-1}+\rmO(|\xi|^2)$. 


%\left(
%\begin{array}{c|c}
% d\ell^2 +\rmO(\ell^4) & \cdots \rmO(\ell^2) \cdots\\ \hline
% \vdots & \raisebox{-15pt}{{\large\mbox{{$I_{n-1}+\rmO(\ell^2)$}}}} \\[-4ex]
%\rmO(\ell^2) & \\[-0.5ex]
%  \vdots &
%\end{array}
%\right)

%Next we claim that $d\neq 0$, recall that the determinant of a block diagonal matrix of the form $\begin{pmatrix}
%A&B\\
%C&D
%\end{pmatrix}$ is given by the formula $\det(D)\det(A-BD^{-1}C)$ provided $D$ is invertible, which is the case for $s$ sufficiently close to $0$. We compute
%\begin{align*}
%\det P\mathcal{T}(s)Q &=\det (D(s))\det[A(s)-B(s)D(s)^{-1}C(s)] \\
%& = \det(D(s)) (A(s)- B(s)D(s)^{-1}C(s)\\
%& =(1+\rmO(s^2)) (ds^2 + o(s^2)) = ds^2 +o(s^2).
%\end{align*}
%By assumption, $\det P\mathcal{T}(s)Q = \det(PQ)\det(\mathcal{T}(s))=\det(PQ)(\mathcal{D}''(0) s^2/2+o(s^2))$ as $s \to 0$, hence $d = \mathcal{D}''(0)\det(PQ)/2$, which is nonzero as claimed.


\paragraph{Step 2.} Set $
H(\xi) = \diag\{\frac{1+|\xi|^2}{|\xi|^2}, I_{k-1}\}$ for $\xi \neq 0$. 
We find
\begin{align*}
P\That(\xi)QH(\xi) &= \left(
\begin{array}{c|c}
  |\xi|^2+o(|\xi|^2) &  \cdots \rmO(|\xi|^2)\cdots \\ \hline
  \vdots & \raisebox{-10pt}{{\large\mbox{{$I_{k-1}+\rmO(|\xi|^2)$}}}} \\[-4ex]
  \rmO(|\xi|^2) & \\[-0.5ex]
  \vdots &
\end{array}\right)\left(
\begin{array}{c|c}
  \frac{1+|\xi|^2}{|\xi|^2} & 0 \cdots 0 \\ \hline
  0 & \raisebox{-10pt}{{\large\mbox{{$I_{k-1}$}}}} \\[-4ex]
  \vdots & \\[-0.5ex]
  0 &
\end{array}\right)\\
&= \left(
\begin{array}{c|c}
  1+|\xi|^2+o(|\xi|^2) &  \cdots \rmO(|\xi|^2)\cdots \\ \hline
  \vdots & \raisebox{-10pt}{{\large\mbox{{$I_{n-1}+\rmO(|\xi|^2)$}}}} \\[-4ex]
  \rmO(1) & \\[-0.5ex]
  \vdots &
\end{array}\right) .
\end{align*} 

Therefore, for $|\xi|$ small, $P\That(\xi)QH(\xi)$ is invertible with uniform bounds on the inverse.

 We then define $\tilde{S}(s)$ and $S(s)$ for all $s\ge 0$ through
\[
\tilde{S}(s)=\begin{cases}
P\mathcal{T}(s)QH(s),\hspace{0.1in} &s\neq 0\\ 
S^0.\hspace{0.1in} &s=0
\end{cases}, \hspace{0.5in}\text{and}\hspace{0.5in} S(s) = \tilde{S}(s)^{-1}.
\]


Next, for $s \neq 0$, $\tilde{S}(\ell) \in \mathscr{C}^2$ since $\mathcal{T} \in \mathscr{C}^2$. Near $s = 0$, we can write $P\mathcal{T}(s)Q = \begin{pmatrix}
s^2\tilde{A}(s)&B(s)\\
s^2\tilde{C}(s)&D(s)
\end{pmatrix}$ such that $\tilde{A},\tilde{C}$ are continuous by Taylor's theorem. As a consequence $\tilde{S}(s)= PT(s)QH(s)$ is continuous near $s = 0$. Since $\tilde{S}^{-1}(0)$ is an invertible matrix, by the implicit function theorem, we conclude that the entries of $S$ are continuous functions in $s$.
 
Hence, if we apply $S(s)$ to $P\mathcal{T}(s)Q$ from the left, it then yields
 \[
 S(s)P\mathcal{T}(s)Q = H(s)^{-1}[P\mathcal{T}(s)Q]^{-1}P\mathcal{T}(s)Q = H(s)^{-1}=\diag\{m(s),I_{k-1}\},
 \]
 which is as stated in the lemma.
 The entries of $S(s)$ defined here are continuous for all $s \ge 0$,  also, by definition, for $s \neq 0$, we have $S(s) = H(s)^{-1}[P\mathcal{T}(s)Q]^{-1} = H(s)^{-1}Q^{-1}\mathcal{T}(s)^{-1}P^{-1}$.
 
By (H1), the Fourier transform $\widehat{\K}$ satisfies $|\widehat{\K}_{i,j}(s)| \to 0$ as $s \to \infty$ for $1\le i,j\le k$. Therefore $\mathcal{T}(s)-I_n=\widehat{\K}(s)$ converges to the $0$ matrix as $s \to 
\infty$. If we set $S^\infty := \diag\{d,I_{n-1}\}Q^{-1}P^{-1}$, which is an invertible matrix as $d\neq 0$, it then follows that $S(s)-S^\infty\to 0$ as $s \to \infty$. Since both $S(0)=S^0$ and $S^\infty$ are invertible, we conclude that $S(s)$ is invertible for all $s$, with uniform bounds on its inverse and the entries of $S$ are bounded continuous functions in $s$. This concludes the proof.
 \end{Proof}
Thus, if we denote by $L$ the multiplier operator with symbol $S(s)$, denote by $M$ the multiplier operator with symbol $m(s)$, set $V(x)=Q^{-1}U(x)$, with standard coordinates $V(x)=(v_c(x),v_2(x),\ldots,v_{n}(x))^T$. Then after precondition \eqref{system} (which is valid since $L,P,Q$ are invertible by above) with $LP$ we obtain an equivalent equation
\begin{equation}\label{TranEq}
\diag(M, I_{n-1})V=LP\Nl(QV;\mu).
\end{equation}

In the next subsection we shall further simplify \eqref{TranEq} by rescaling the variables, so that its behavior as $\mu \to 0$ will be revealed.


\subsection{Change of coordinates and rescaling }
% If $\mathcal{E}$ denotes the vector which spans the kernel of $\mathcal{T}(0)=I_n+\widehat{\K}(0)$, let $\mathcal{E}_2,\ldots,\mathcal{E}_{n-1}$ be a set of basis which spans the complement of $\ker \mathcal{T}(0)$, set $\mathcal{F}_i = \mathcal{T}(0)\mathcal{E}_i$, choose $\mathcal{E}_1^*$ such that $\langle \mathcal{E}_1^*,\mathcal{E}_1^*\rangle=1$ and $\langle \mathcal{E}_1^*, \mathcal{F}_i\rangle=0$ where $\langle \cdot,\cdot \rangle$ is the usual scalar product in $\R^n$.




%\paragraph{Derivative of $\tilde{\Nl}$}
%then we have $\frac{\partial^2}{\partial \mu \partial v_c}\tilde{\Nl}_c(0;0)$ is given by $\langle \mathcal{E}_1^*, D_{\mu,U} \Nl(0;0) \mathcal{E}_1\rangle$, and the $n-1$ components of $\frac{\partial^2}{\partial\mu \partial v_c}\tilde{\Nl}_h(0;0)$ is given by $\langle\mathcal{F}_i, D_{\mu,U} \Nl(0;0)\mathcal{E}_1\rangle$ for $i=2,\ldots,n-1$. Similarly, the second derivative in $U$ can be calculated as: $\frac{\partial^2}{\partial v_c^2}\tilde{\Nl_c}(0;0)=\langle\mathcal{E}_1^*, D_{U,U}\Nl(0;0)\mathcal{E}_1\rangle$ and $\frac{\partial^2}{\partial v_c^2}\tilde{\Nl_h}(0;0)=\langle\mathcal{F}_i, D_{U,U}\Nl(0;0)[\mathcal{E}_1,\mathcal{E}_1]\rangle$ for $i=2,\ldots,n-1$.


\paragraph{Transformed equations.} 
Write $v_h = (v_2,\ldots,v_{k})^T$ in the standard coordinate in $\R^{k-1}$. Set $\mathcal{H}(V;\mu) :=P\Nl(QV;\mu)$. Then with respect to the standard basis in $\R^k$, we define $\mathcal{H}_c(V;\mu)$ to be the first component of the nonlinearity $\mathcal{H}$ and $\mathcal{H}_h(V;\mu)$ to be the remaining $k-1$ components of $\mathcal{H}$. Then, also with respect to the usual basis, we write the matrix $S$ and the multiplier operator $L$ which has symbol $S$ in components 
\[
S(s) = \begin{pmatrix}
S_{cc}(s) & S_{ch}(s)\\
S_{hc}(s) & S_{hh}(s) 
\end{pmatrix}, \hspace{0.2in}
L  = \begin{pmatrix}
L_{cc} & L_{ch} \\
L_{hc} & L_{hh} 
\end{pmatrix},
\]
where terms with subscript $cc$ denote a scalar, terms with subscript $ch$ a $(k-1)$ dimensional row vector, terms with subscript $hc$ a $(k-1)$ dimensional column vector, terms with subscript $hh$ a $(k-1)\times (k-1)$ matrix.

In this notation, system \eqref{system} becomes
\begin{align}
M v_c + L_{cc}\mathcal{H}_c(v_c,v_h;\mu) + L_{ch}\mathcal{H}_h(v_c,v_h;\mu)= 0\label{exeqnu0},\\
v_h +  L_{hc}\mathcal{H}_c(v_c,v_h;\mu) + L_{hh}\mathcal{H}_h(v_c,v_h;\mu) = 0 \label{exeqnuh}.
\end{align}


By part $(i)$ of Hypothesis (H2), we may use Taylor's theorem to write $\mathcal{H}_j$ with $j=c,h$ as
\begin{align*}
\mathcal{H}_j(v_c,v_h;\mu) &=\left( a^j_{101} \mu v_c+a^j_{011}\mu v_h+a^j_{110}v_cv_h + a^j_{200}v_c^2+a^j_{020}[v_h,v_h] \right)+ \Rm_j(v_c,v_h;\mu)\\
&:= \B_j(v_c,v_h;\mu)+\Rm_j(v_c,v_h;\mu),
\end{align*}
where $a^j_{lmn} = \frac{\partial^2 \mathcal{H}_j}{\partial v_c^l \partial v_h^m \partial \mu^n}(0,0;0), l+m+n=2, 0\le l,m,n\le 2$ are the partial derivatives of order $2$ at $(v_c,v_h;\mu)=(0,0;0)$, and the remainder $R_j$ satisfies the pointwise estimate
\begin{equation}\label{odR}
|R_j(v_c,v_h;\mu)| = |R_j(V;\mu)| = \rmO(\mu^2|V|+\mu|V|^2+|V|^3)
\end{equation} 


%\rmO(v_c^2|v_h|,v_c|v_h|^2,v_c^3,|v_h|^3,\mu v_c^2, \mu |v_h|^2,\mu v_c|v_h|, \mu^2v_c, \mu^2|v_h|),
for $(V;\mu)$ bounded.

We are in particular interested in the coefficients of the terms $\mu v_c$ and $v_c^2$. In equation \eqref{exeqnu0}, the coefficient of $\mu v_c$ is given by $L_{cc}a_{101}^c+L_{ch}a_{101}^h$, and the coefficient of $v_c^2$ is given by $L_{cc}a_{200}^c+L_{ch}a_{200}^h$. Using hypothesis (H2), we claim that
\[
\alpha = a_{101}^c =\widehat{L}_{cc}(0)a_{101}^c+\widehat{L}_{ch}(0)a_{101}^h , \hspace{0.1in}\text{ and }\hspace{0.1in}
\beta =a_{200}^c =\widehat{L}_{cc}(0)a_{200}^c+\widehat{L}_{ch}(0)a_{200}^h .
\]

Indeed, to verify the first assertion, recall that by definition we have $\widehat{L}(0) = S^0$. In particular, 
$\widehat{L}_{cc}(0)=1$ and $\widehat{L}_{ch}(0)=(0,\ldots,0)$, thus verifying the second equality $a_{101}^c=\widehat{L}_{cc}(0)a_{101}^c+\widehat{L}_{ch}a_{101}^h$. 
To compute the coefficient of $\mu v_c$ in terms of the 
original function $\Nl$, let $e_1$ denote the standard coordinate vector $(1,0,\ldots,0)^T$, then the derivative 
$a_{101}^c=\dfrac{\partial^2}{\partial \mu \partial v_c}  \mathcal{H}_c(0,0;0)$ is given by
\[
\langle D_{\mu,V}\mathcal{H}(0;0)e_1,e_1\rangle=\langle D_{\mu,U} P\Nl(0;0)Q e_1,e_1\rangle = \langle PD_{\mu,U}\Nl(0;0) \mathcal{E}_1,e_1\rangle  = \langle D_{\mu,U}\Nl(0;0)\mathcal{E}_1,\mathcal{E}_1^*\rangle = \alpha,
\]
which verifies the first equality $\alpha=a_{101}^c$. The computations for $\beta$ is similar.

In the next paragraph, we shall make a series of rescalings to simplify the equation.

\paragraph{Rescaling.} Define new variables $\tilde{v}_c,\tilde{v}_h, \tilde{\mu}$ through 
\[
\mu=\frac{1}{\alpha}\tilde{\mu}, \hspace{0.1in} v_c(\cdot)=\frac{-1}{\beta}\tilde{\mu}\tilde{v}_c(\sqrt{\tilde{\mu}} \cdot), \hspace{0.1in} v_h(\cdot)=\tilde{\mu} \tilde{v}_h(\sqrt{\tilde{\mu}} \cdot),
\]
since we assumed $\alpha >0$, $\tilde{\mu}$ is positive, and we shall write $\eps := \sqrt{\tilde{\mu}}$.

We substitute these variables in equation \eqref{exeqnu0} and \eqref{exeqnuh}, divide the first equation by $(-1/\beta)\eps^4$, the second by $\eps^2$, and then obtain
\begin{align}
\eps^{-2}M^\eps \tilde{v}_c + \sum_{j=c,h} L_{cj}^{\eps}[\tilde{\B}_j(\tilde{v}_c,\tilde{v}_h)+\eps^{-4}\tilde{\Rm}_j(\tilde{v}_c,\tilde{v}_h;\eps)],\label{rseqnu0}\\
\tilde{v}_h +\sum_{j=c,h} L_{hj}^{\eps}[\eps^2\tilde{\B}_j(\tilde{v}_c,\tilde{v}_h)+\eps^{-2}\tilde{\Rm}_j(\tilde{v}_c,\tilde{v}_h;\eps)] = 0. \label{rseqnuh}
\end{align}
Note that here equation \eqref{rseqnu0} and \eqref{rseqnuh} holds pointwise in $y = \sqrt{\tilde{\mu}} x$. Since $x$ is arbitrary, they hold for all $y \in \R^n$. We rather view them as functional equation in $\tilde{v}_c(\cdot)$ and $\tilde{v}_h(\cdot)$.

The rescaled linear operators $M^\eps $ and $L^\eps_j$ for $j = cc,ch,hc,hh$ have symbols $m(\eps s)$, $S_j(\eps s)$ respectively. 

The rescaled nonlinear terms $\tilde{\B_j}, \tilde{\Rm_j}$ for $j=c,h$ are defined through 
\begin{align*}
\tilde{\B}_j(u,v)&=\frac{a^j_{101}}{\alpha}  u+\frac{a^j_{011}}{\alpha} v+a^j_{110}uv + \frac{a^j_{200}}{-\beta}u^2+a^j_{020}(-\beta)v^2 ,\\
\tilde{R}_j ( u,v;\eps)&=\Rm_j \left(\frac{\eps^2u}{-\beta},\eps^2v;\frac{\eps^2}{\alpha}\right).
\end{align*}

In particular, the coefficient of the term $\tilde{v}_c$ now equals $ a_{101}^c/\alpha=1$, and the coefficient of $\tilde{v}_c^2$ now equals $a_{200}^h/(-\beta)=\beta/(-\beta)=-1$ by the computation we have done in the previous step. As a consequence, we have 
\[
\tilde{B}_c(\tilde{v}_c,\tilde{v}_h) = \tilde{v}_c-\tilde{v}_c^2 + \rmO(|\tilde{v}_h|+|\tilde{v} _h|^2+\tilde{v}_c|\tilde{v} _h|).
\] 



From the multiplication algebra property and Sobolev embedding of $H^\ell(\R^n;\R^k)$ with $\ell>n/2$, for any $u\in H^\ell(\R^n), v\in H^\ell(\R^n;\R^{k-1})$, and $j=c,h$ we have the following estimates

\begin{equation}\label{Blest}
\|\tilde{\B}_j(u,v)\|_{H^\ell} \le C\left(\|u\|_{H^\ell}+\|v\|_{H^\ell}+\|u\|_{H^\ell}\|v\|_{H^\ell} +\|u\|_{H^\ell}^2+\|v\|^2_{H^\ell}\right),
\end{equation}
with some constant $C$. For the remainder terms $\tilde{\Rm}_j$, we have (see appendix for proof)
\begin{equation}
 \|\tilde{\Rm}_j(u,v;\eps)\|_{H^\ell} = \rmO(\eps^6)  \label{Nlest}
\end{equation}
as $\eps \to 0$ from \eqref{odR}.

 For the linear operator $L^\eps$, we have the following estimate (also see appendix for proof)
\begin{equation}\label{Liest}
\lim_{\eps \to 0} \|(L^\eps - S^0)u \|_{H^\ell} = 0,
\end{equation}
if $u \in H^\ell(\R^n;\R^k)$. As a result, the components of $L^\eps$, which are  linear operators $L^\eps_{cc} : H^\ell(\R^n) \to H^\ell (\R^n), L^\eps_{ch}:H^2(\R^n;\R^{k-1}) \to H^\ell (\R^n), L^\eps_{hc}:H^\ell (\R^n) \to H^\ell(\R^n;\R^{k-1}),$ and $L^\eps_{hh}:H^\ell(\R^n;\R^{k-1}) \to H^\ell(\R^n;\R^{k-1})$ are bounded with uniform bounds in $\eps$.


In particular, $L_{cc}^\eps \to 1$ and $L_{ch}^\eps \to (0,\ldots,0)$ as $\eps \to 0$. Hence, it remains to understand the behavior of the term $\eps^{-2}M^\eps v_c$ as $\eps \to 0$. But before that, we first reduce \eqref{rseqnu0} and \eqref{rseqnuh} to a scalar equation using a fixed point argument in the next subsection. The main result will be proved shortly after.

To further ease notations, we drop the tildes, and still use $v_j,\B_j,\Rm_j$ ($j=c,h)$ for the same variables after the rescaling. 



\subsection{Lyapunov-Schmidt reduction and proof of the main result}
We first solve \eqref{rseqnuh} to obtain $v_h$ as a function of $v_c$ by a fixed point argument. We then substitute this function back into equation \eqref{rseqnu0} to obtain a scalar equation for $v_c$ and $\eps$, which will be solved again using a fixed point argument.

We write the left hand side of \eqref{rseqnuh} as $\G(v_h; v_c,\eps)$ with $\G$ defined so that
\[
\G(v;u,\eps) = v+ \sum_{j=c,h} L_{hj}^{\eps}\left( \eps^2\B_j(u,v;\eps)+\eps^{-2}\Rm_j(u,v;\eps) \right), 
\]
using estimates \eqref{Blest} and \eqref{Nlest}, we have $\G : H^\ell(\R^n;\R^{k-1}) \times H^\ell(\R^n) \to H^\ell(\R^n;\R^{k-1})$ for each $\eps >0$.
Note that we are treating $v_c$ as an additional (Banach space-valued) parameter. The following lemma accomplishes what we were planning to do.
\begin{lemma}\label{Lemuh} Fix $r>0$ not necessarily small, let $B_r$ denote the ball centered at $0$ with radius $r$ in $H^\ell(\R^n)$, there then exist $\eps_0>0$ sufficiently small and a map $\psi(u,\eps): B_r \times (-\eps_0,\eps_0) \to H^\ell(\R^n;\R^{k-1})$ such that $v = \psi(u, \eps)$ solves $\G(v;u,\eps) = 0$. Moreover, the map $u \mapsto \psi(u,\eps)$ is smooth for $u\in B_r$, and we have 
\[
\|\psi(u,\eps)\|_{H^\ell} = \rmO(\eps^2), \hspace{0.1in}\|D_u\psi(u,\eps)\|_{H^\ell \to H^\ell} = \rmO(\eps^2),
\] as 
$\eps \to 0$, uniformly for $u\in B_r$ where $D_u\psi(u,\eps)$ denotes the Frechet derivative of $\psi$ with respect to $u$ at the point $(u,\eps)$.  \end{lemma}
\begin{Proof}We will solve $\G(v;u,\eps)=0$ using a Newton iteration scheme. For $u \in B_r$ and $\eps_0$ small, we claim the following properties hold for $\G$:
\begin{enumerate}
\item $\|\G(0;u,\eps)\|_{H^\ell} = \rmO(\eps^2),$ uniformly in $u\in B_r$ and $|\eps| < \eps_0$.
\item $\G$ is smooth in $v$, and $D_v \G(0; u, \eps):H^\ell(\R^n;\R^{k-1}) \to H^\ell(\R^n;\R^{k-1})$ is bounded invertible with uniform bounds on the inverse for $|\eps|<\eps_0$ and $u \in B_r$. 
\end{enumerate}

For $(i)$, since $L^\eps$ is uniformly bounded in $\eps$, there exist a constant $K$ such that $\| L_{hc}^\eps\|_{H^\ell \to H^\ell}+\| L_{hh}^\eps \|_{H^\ell \to H^\ell } \le K$, we then have
\begin{align*}
\|\G(0,u;\eps)\|_{H^\ell} &\le \eps^2K(\|\mathcal{B}_c(u,0;\eps)\|_{H^\ell} +\|\mathcal{B}_h(u,0;\eps)\|_{H^\ell})\\
&+\eps^{-4}K(\|\Rm_c(u,0;\eps)\|_{H^\ell}+\|\Rm_h(u,0;\eps)\|_{H^\ell}).
\end{align*}
Using estimates \eqref{Blest} and \eqref{Nlest}, we have $\|\G(0;u,\eps)\|_{H^\ell} \le C(r) \eps^2$ uniformly in $u\in B_r$ and $\eps$ small.

For $(ii)$, we conclude the smoothness of $\mathcal{G}$ in $v$ using Lemma \ref{spHest} in the Appendix and the fact that $L^\eps_j$ are bounded linear operators. We compute the Frechet derivative of $\G$ to obtain
\[ 
D_v\G(v;u,\eps) w = w+ \sum_{j=c,h} L_{hj}^\eps (\eps^2 D_v \mathcal{B}_j (u,v;\eps)+ \eps^{-2} D_v\Rm_j(u,v;\eps) ) w
\] 
for $w \in H^\ell(\R^n;\R^{k-1})$. Using estimate \eqref{Nlest}, we see that $D_v\G(0; u, \eps)$ is an $\rmO(\eps^2)$ perturbation of the identity as an operator on $H^\ell(\R^n;\R^{k-1})$ uniformly for $u \in B_r$. Thus, if $\eps_0$ is small enough, then for all $\eps$ with $|\eps|<\eps_0$, we have that $D_v\G(0;u,\eps)$ is bounded invertible with uniform bounds in $\eps$.


After establishing $(i)$ and $(ii)$, fix $\delta>0$ and $u \in B_r$. Let $N_\delta$ denote the closed ball of radius $\delta$ around $0$ in $H^\ell(\R^n;\R^{k-1})$, we introduce a map $\cS(\cdot; u,\eps): H^\ell(\R^n;\R^{k-1}) \to H^\ell(\R^n;\R^{k-1})$ as follows
\[
\cS(v; u,\eps) = v - D_v\G(0;u, \eps)^{-1}[\G(v;u,\eps)].
\]
We then find
\[
\|\cS(0;u,\eps) \|_{H^\ell} \le \|D_v\G(0;u,\eps)^{-1}\|_{H^\ell\to H^\ell} \|\G(0;u, \eps)\|_{H^\ell} = \rmO(\eps^2).
\]

Also, $D_v\cS(0;u,\eps) = 0$ by definition, and $\cS$ is smooth in $v$ by $(ii)$. Therefore, if $\delta$ is small and $v\in N_\delta$, it then follows that $\|D_vS(v;u,\eps)\|_{H^\ell \to H^\ell} \le C\delta$ for some constant $C$ independent of $\delta$.

Then we start our iteration with $v_0 = 0$, $v_{n+1} = \cS(v_n;u,\eps)$, $n\ge 0$. Suppose by induction $v_k \in N_\delta$ for $1\le k \le n$, then
\[
\|v_{n+1}-v_n\|_{H^\ell} \le C\delta\|v_n-v_{n-1}\|_{H^\ell},
\]
by the mean value theorem. Therefore
\[
\|v_{n+1}\|_{H^\ell} \le \frac{C}{1-C\delta}\|v_1-v_0\|_{H^\ell} = \frac{C}{1-C\delta}\|\cS(0;u,\eps)\|_{H^\ell}.
\]
This implies that for $\eps$ small and $u \in B_r$, we have $v_{n+1} \in N_\delta$, and that $\cS$ is a contraction for $\delta$ sufficiently small. Then, as in the proof of Banach's fixed point theorem, we conclude that $v_n \to v = \psi(u,\eps)$ as $n\to \infty$ and $v$ is a fixed point of $\cS$. Note that we automatically get $\|\psi(u,\eps)\|_{H^\ell} = \rmO(\eps^2)$ uniformly in $u\in B_r$. 

To show the smooth dependence of $\psi(u,\eps)$, we note that $\G(v;u,\eps)$ is also smooth in $u$ by Lemma \ref{spHest}. By choosing $\eps$ small, the contraction constant for $\cS$ can be chosen uniformly in $u \in B_r$. Hence by adopting the proof of the uniform contraction principle (see \cite{chicone2006ordinary}, Theorem 1.244), we conclude that $\psi$ depends smoothly on $u$ as well.

Finally, to get the estimate $\|D_u\psi\|_{H^\ell \to H^\ell} = \rmO(\eps^2)$, we differentiate the equation $0 = \G(\psi(u,\eps);u,\eps)$ in $u$ for $u\in B_r$ to see $D_u\psi$ satisfies the equation
\[
D_v\G(\psi(u,\eps); u,\eps)  D_u\psi(u,\eps) + D_u\G(\psi(u,\eps);u,\eps) = 0.
\]

Now, $D_u\G(v;u,\eps)$ is of the form
\[
D_u\G(v;u,\eps) w =  \sum_{j=c,h} L_{hj}^\eps (\eps^2 D_u \mathcal{B}_j(u,v;\eps) + \eps^{-2} D_u\Rm_j(u,v;\eps) ) w,
\]
hence, for $u \in B_r$ and $v=\psi(u,\eps) \in N_\delta$, we have $\|D_u\G(v;u,\eps)\|_{H^\ell \to H^\ell} = \rmO(\eps^2)$ again by estimate \eqref{Blest} and \eqref{Nlest}.

On the other hand, $D_v\G(v;u,\eps)$ is uniformly invertible in $\eps$ for $v=\psi(u,\eps)\in N_\delta$ and $u\in B_r$ as calculated previously. Therefore we can write $D_u\psi(u,\eps) = -[D_v\G]^{-1} D_u\G $ and conclude that 
\[
\|D_u\psi(u,\eps)\|_{H^\ell \to H^\ell} \le C(r,\delta)\eps^2.
\] This finishes the proof.
\end{Proof}

\begin{remark} Because of the dependence of the convolution operator $L_{hc}^\eps, L_{hh}^\eps$ on $\eps$ is not smooth at $\eps = 0$, we cannot use the usual implicit function theorem directly to solve the equation $\G(v;u,\eps) = 0$. We follow the Newton iteration scheme as in \citep{faye2013existence} to circumvent this problem.
\end{remark}

Using lemma \ref{Lemuh}, we substitute $v_h = \psi(v_c,\eps)$ into equation \eqref{rseqnu0}. We obtain the following scalar equation

\begin{equation} \label{1dnl}
0 = \eps^{-2}M^\eps v_c + \sum_{j=c,h}L_{cj}^\eps\left[B_j(v_c,\psi(v_c,\eps))+\eps^{-4}\Rm_j(v_c,\psi(v_c,\eps);\eps)\right].
\end{equation}


It is now crucial to understand the behavior of the operator $M^\eps$ as $\eps \to 0$. Recall that by definition
\[
\widehat{M^\eps v}(\xi) = m(\eps \xi)\widehat{v}(\xi) = \frac{d|\eps\xi|^2}{1+|\eps\xi|^2} \widehat{v}(\xi), 
\]
for any $v\in H^\ell(\R^n;\R^k)$. We then define a new operator $\mathcal{M}^\eps$ through 
\[ 
\widehat{\mathcal{M}^\eps v}(\xi) = \frac{m(\eps\xi)}{|\eps\xi|^2}\widehat{v}(\xi)=\frac{d}{1+|\eps\xi|^2} \widehat{v}(\xi). 
\] 
Since $d/(1+|\eps \xi|^2)$ is a bounded function on $\R^n$, $\M^\eps$ maps $H^\ell(\R^n;\R^k)$ into itself. 

For $v\in H^\ell(\R^n,\R^k)$, $(\M^{\eps})^{-1}$ is defined through
\[
\widehat{(\M^{\eps})^{-1}v} (\xi) = \frac{|\eps\xi|^2}{m(\eps\xi)} \widehat{v}(\xi)= \frac{1+|\eps \xi|^2}{d} \widehat{v}(\xi),
\]
moreover we have:
\begin{align*}
\|((\M^\eps)^{-1}-d^{-1})v\|_{H^{\ell-2}} &=\left\| \left(\frac{1+|\eps\xi|^2}{d}-d^{-1}\right)\widehat{v}(\xi)(1+|\xi|^2)^{\frac{\ell-2}{2}}\right\|_{L^2} 
\\
& \le \sup_{\ell} \left|\frac{|\eps\xi|^2}{1+|\xi|^2}\right| \|\widehat{v}(\xi)(1+\xi^2)^{\frac{\ell}{2}} \|_{L^{2}}\\ 
&\le \eps^2 \|v\|_{H^\ell}.
\end{align*}

Therefore, considered as an operator from $H^\ell (\R^n;\R^k)$ to $H^{\ell-2}(\R^n;\R^k)$, $(\M^\eps)^{-1}$ is well-defined, and $\|(\M^{\eps})^{-1}v - d^{-1}v\|_{H^{\ell-2}} \to 0$ as $\eps \to 0$ for $v \in H^\ell (\R^n;\R^k)$. This simple observation is central to identifying the leading-order terms and  we state it as a lemma.


\begin{lemma}\label{estmult}The multiplier operator $(\M^\eps)^{-1}$ with symbol $\dfrac{|\eps\xi|^2}{m(\eps\ell)}$ is well defined, maps from $H^\ell  (\R^n;\R^k)$ into $H^{\ell-2} (\R^n;\R^k)$, and satisfies the estimate
\[
\|(\M^\eps)^{\-1}-d^{-1}\|_{H^\ell \to H^{\ell-2}} = \rmO(\eps^2).
\]
\end{lemma}
Our main result, Theorem \ref{MainRes}, will be proved by the following proposition, which solves the reduced bifurcation equation \eqref{1dnl} from the Lyapunov-Schmidt reduction in lemma \ref{Lemuh}.


\begin{proposition}\label{prop}Assume $d>0, n<6$ and $\ell>n/2+1$, let $v_*$ be the unique  solution of the equation $d\Delta v - v +v^2 = 0$. If $\eps_1>0$ is sufficiently small, then for $0<|\eps|<\eps_1$, there exist a family of solutions to \eqref{1dnl} of the form $v_c(\cdot;\eps) = v_*(\cdot)+w(\cdot; \eps)$. Here $w=w(\cdot,\eps) \in H^{\ell}_{r}(\R^n;\R^k)$ is a family of correctors parametrised by $\eps $ such that $\|w(\cdot,\eps)\|_{H^\ell} \to 0$ as $\eps \to 0$.
\end{proposition}

%First, by assumptions on $f$, the Taylor expansion for $f$ near $(u,\mu)=(0,0)$ is
%\[
%f(u,\mu) = A u\mu - B u^2 + O(\mu^2 u, \mu u^2, u^3)
%\]
%where $A = f_{u\mu}(0,0), B=-f_{uu}(0,0)$, for definiteness we assume here that $A,B>0$, then we rescale $u$ and $\mu$ by $u \mapsto su$ and $\mu \mapsto t\mu$, where $s,t$ are constants to be chosen, then we have
%\[
%-su + sK\ast u = Ast u\mu -Bs^2u^2 + O(\mu^2 u, \mu u^2,u^3)
%\]
%cancel out $s$, we see choose $s=1/B$ and $A=1/t$ lead to the following equation for $u$ %and $\mu$:
%\[
%-u+K\ast u=f(u; \mu) = \mu u - u^2 + O(\mu u^2,\mu^2u,u^3),
%\] 
%then we rescale $u(x) = \mu v(\sqrt{\mu}x)$, we have an equation in $v$: 
%\begin{equation}  \label{scl nl}
%-v(y) + K_\eps \ast v (y) = \eps^2(v-v^2)+O(\eps^4v)
%where $\eps^2 = \mu$ and $K_\eps = \eps^{-1}K(\cdot/\eps)$, and $y =\eps x$. From this point we focus on $\eqref{scl nl}$.

%Dividing by $\eps^2$, and take Fourier transform of both sides of the equation (we assume we are solving in $H^2(\R)$.)

%\[
%\frac{-1+\hat{K}_\eps(\ell)}{-\eps^2\ell^2}(-\ell^2)\hat{v}(\ell) = \widehat{v-v^2}(\ell) %+ O(\eps^2 \hat{v})
%\] 

%We define the operator $M_\eps$ so that the Fourier multiplier $\widehat{M}_\eps(\ell) = \frac{-1+\hat{K}_\eps(\ell)}{-\eps^2\ell^2}= \frac{-1+\hat{K}(\eps\ell)}{-\eps^2\ell^2}$, by assumptions on $K$, we 
%know $M_\eps$ is bounded as an operator from $L^2$ to $L^2$, but $\sup |\widehat{M_\eps}(\ell)|$ does not necessarily go to zero as $\eps \to 0$.

%We take inverse Fourier transform and get back the equation in physical space:
%\begin{equation}\label{eq phy}
%M_\eps v''(y) = v(y)-v^2(y) + O(\eps^2 v)
%\end{equation}

\begin{Proof}
We substitute the ansatz $v_c = v_* + w$ into \eqref{1dnl}, where $v_*$ is as stated in the lemma and $w \in H^2$. We will determine an equation for $w$ and $\eps$ and show that it can be solved using Newton iteration scheme near $(w,\eps)=(0,0)$.
First, for the term $\eps^{-2}M^\eps v_c$ with $v_c \in H^\ell$, we apply Fourier transform to obtain
\[
\eps^{-2}m(\eps\xi)\widehat{v}_c(\xi) = -\frac{m(\eps\xi)}{|\eps\xi|^2}(-|\xi|^2)\widehat{v}_c(\xi) = -\widehat{\M^\eps \Delta v_c},
\]
and equation \eqref{1dnl} becomes
\[
0 = -\M^\eps \Delta v_c + \left(L_{cc}^\eps a_{101}^c+L_{ch}^\eps a_{101}^h\right)v_c+\left(L_{cc}^\eps a_{110}^c+L_{ch}^\eps a_{110}^h\right)v_c^2 + \Rm(v_c,\psi;\eps),
\]
where $\Rm(v_c,\psi;\eps)$ contains all the terms of order $\eps^2$ and higher,
\[
\Rm(v_c,\psi;\eps) =\sum_{j=c,h} L_{cj}^\eps\left[ \frac{a_{011}^j}{\alpha}\psi+a_{110}^j v_c\psi+a_{020}^j (-\beta)[\psi,\psi]+\eps^{-4}\Rm_j(v_c,\psi;\eps)\right].
\]
Indeed, for $w$ with $v_*+w \in B_r$, we claim that $\Rm$ satisfies the estimate $\|\Rm\|_{H^\ell} = \rmO(\eps^2)$.  To see this, we first apply Lemma \ref{Lemuh} with $r = 2\|v_*\|_{H^\ell}$ to obtain $\psi = \psi(v_*+w,\eps)$ which satisfies $\|\psi(v_*+w,\eps)\|_{H^\ell} = \rmO(\eps^2)$.

The linear operators $L_{cc}^\eps, L_{ch}^\eps$ are uniformly bounded in $\eps$, so that we have
\[
\left\|\sum_{j=c,h} L_{cj}^\eps\left( \frac{a_{011}^j}{\alpha}\psi+a_{110}^j v_c\psi+a_{020}^j (-\beta)[\psi,\psi]\right)\right\|_{H^\ell} \le K(\|\psi\|_{H^\ell}+\|\psi\|_{H^\ell}^2) = \rmO(\eps^2),
\]
for some constant $K$ from Lemma \ref{Lemuh}.

On the other hand, the remainders $\Rm_c$ and $\Rm_h$ satisfy $\|\Rm_c\|_{H^\ell}= \rmO(\eps^6), \|\Rm_h\|_{H^\ell} = \rmO(\eps^6)$ uniformly for $v_*$ and $w$ such that $v_* +w \in B_r$ as $\eps \to 0$ by  estimates \eqref{Blest} and \eqref{Nlest}. Therefore we conclude that $\|\Rm(v_c,\psi;\eps)\|_{H^\ell} = \rmO(\eps^2)$ for $v_c=v_*+w \in B_r$.
 
Next, add the equation $d\Delta v_*-v_*+v_*^2 =0$ to the right hand side of \eqref{1dnl} and precondition with the operator $(\M^{\eps})^{-1}$. Set $\alpha^\eps = L_{cc}^\eps + \frac{a_{101}^h}{\alpha} L_{ch}^\eps , \beta^\eps=-L_{cc}^\eps +\frac{a_{200}^h}{-\beta} L_{ch}^\eps $ and we find
\begin{eqnarray}
0 &=(\M^\eps)^{-1}\left[ (d-\M^\eps)\Delta v_* -\M^\eps \Delta w+\alpha^\eps(v_*+w)-v_*+\beta^\eps(v_*+w)^2+v_*^2 + \Rm \right] \nonumber \\ 
&= [(\M^{\eps})^{-1}-d^{-1}]\M^\eps d\Delta v_*+(\M^{\eps})^{-1}\left[ (\alpha^\eps-1)v_*+(\beta^\eps+1)v_*^2+\Rm \right]+ \nonumber \\
&-\Delta w+(\M^{\eps})^{-1}\left[\alpha^\eps w+\beta^\eps(2v_*w+w^2)\right], \nonumber \\
&:= F_1(w;\eps)+F_2(w;\eps):= F(w;\eps).  \label{splfynl}
\end{eqnarray}

By Lemma \ref{estmult}, we have that $F$ maps $H^\ell(\R^n)$ to $H^{\ell-2}(\R^n)$. Our goal is to set up a Newton iteration scheme to solve $ F(w,\eps) =0$ for $w$ in terms of $\eps$ as a fixed point problem.

Following the strategy of Lemma \ref{Lemuh}, we shall show
\begin{enumerate}
\item $\|F(0,\eps)\|_{H^{\ell-2}} \to 0$ as $\eps \to 0$.
\item $F(w,\eps)$ is continuously differentiable in $w$ and $D_wF(0,\eps): H^{\ell}_{r}(\R^n) \to H^{\ell-2}_{r}(\R^n)$ is uniformly invertible in $\eps$.
\end{enumerate}
For $(i)$, we note that
\[
F(0,\eps) = F_2(0;\eps) = [(\M^{\eps})^{-1}-d^{-1}]\M^\eps d \Delta v_*+(\M^\eps)^{-1}[(\alpha^\eps-1)v_*+(\beta^\eps+1)v_*^2+\Rm(v_*,\psi;\eps)].
\]

By Lemma \ref{GS} of the Appendix, $\Delta v_* \in H^\ell(\R^n)$ for all $\ell$, since $\M^\eps$ take $H^\ell(\R^n)$ into itself and is uniformly bounded in $\eps$, we conclude from Lemma \ref{estmult} that 
$\|[(\M^{\eps})^{-1}-d^{-1}]\M^\eps d \Delta v_* \|_{H^\ell} \to 0$
as $\eps \to 0$.

Moreover, by Lemma \ref{Lest}, it holds that $\| \alpha^\eps v -v\|_{H^\ell} \to 0$ and $\| \beta^\eps v + v\|_{H^\ell} \to 0$ as $\eps \to 0$ for any $v \in H^\ell(\R^n)$, and the remainder $\Rm(v_*,\psi;\eps)$ satisfies $\|\Rm\|_{H^2} = \rmO(\eps^2)$ as proved earlier. Hence, we conclude that 
\[
\| F(0;\eps)\|_{H^\ell-2} = \| F_2(0;\eps)\|_{H^{\ell-2} }\to 0
\]
as $\eps \to 0$, which proves $(i)$.

For $(ii)$, we first verify that $F$ is continuously differentiable in $w$ from $H^\ell (\R^n)$ to $H^{\ell-2}(\R^n)$. Indeed, take $h, w_0\in H^\ell(\R^n)$ with $w_0$ fixed. We observe that $D_wF(w_0;\eps)h:H^\ell (\R^n) \to H^{\ell-2}(\R^n)$ is given by
\[
D_wF(w_0;\eps)h = -\Delta h+(\M^\eps)^{-1}\left[(a^\eps h)+2v_*\beta^\eps h + 2w_0h)+D_w\Rm h\right].
\]
The smooth dependence on $w_0$ comes from Proposition \ref{spHest}.

Now, at $w_0 = 0$, we see that, $D_wF(0;\eps)h \to -\Delta h+d^{-1}[h-2v_*h]$ in $H^{\ell-2}(\R^n)$ as $\eps \to 0$ for $h \in H^\ell$ because $\|D_w\Rm h\|_{H^\ell} = \rmO(\eps^2)$ as remarked earlier. By lemma \ref{Nondegen} in the Appendix, the operator $\cJ : H^\ell_r(\R^n) \to H^{\ell-2}_r(\R^n)$ given by 
\[
\cJ h = -d \Delta h+h-2v_*h,
\] is bounded invertible. We notice that $D_wF(0;\eps)$ is a small perturbation of $d^{-1}\cJ$, therefore invertible with uniform bounds on the inverse for $\eps$ small enough. This shows $(ii)$.

We now set up the Newton iteration scheme, define $\tilde{\cS}$ through
\[\tilde{\cS}(w;\eps) = w-D_wF(0;\eps)^{-1}[F(w;\eps)].\]

Note that $\tilde{\cS}$ preserves radial symmetry: $\tilde{\cS} : H^\ell_r(\R^n) \to H^{\ell-2}_r(\R^n)$. Therefore we can proceed as in Lemma \ref{Lemuh} to obtain $w=w(\eps)$ which solves $F(w(\eps);\eps) = 0$ for $\eps $ small enough and satisfies $\|w(\eps)\|_{H^\ell} \to 0$ as $\eps \to 0$.
\end{Proof}

Finally, we prove Theorem \ref{MainRes}.
\begin{Proof}[ of Theorem 2.1.] We now write the tildes for the rescaled variables. From proposition \ref{prop}, we know that \eqref{1dnl} has a solution of the form $\tilde{v}_c(\cdot) = v_*(\cdot)+w(\cdot;\eps)$. Together with $\tilde{v}_h = \psi(\tilde{v}_c,\eps)$, reverting the rescaling, we obtain $v_c(\cdot) = -\frac{\alpha}{\beta}\mu \tilde{v}_c(\sqrt{\alpha\mu }\cdot)$ and $v_h(\cdot) = \alpha\mu \tilde{v}_h(\sqrt{\alpha\mu}\cdot)$ as solutions to \eqref{exeqnu0} and \eqref{exeqnuh}.

 Now, recall that $V=(v_c,v_h)^T$ and the original variable $U$ are related by $U= QV$ where $Q$ is defined in the proof of Lemma \ref{Lem1}. We conclude that $U(\cdot)=v_c(\cdot)\mathcal{E}_0+v_{\perp}(\cdot)$, where $v_{\perp}$ takes values in the complement of $\mathcal{E}_0$. The behavior of $v_c,v_{\perp}$ as $\mu \to 0$ is a direct consequence of Lemma \ref{Lemuh} and Proposition \ref{prop}.
\end{Proof}
\section{Discussion}
In this section we make remarks on our results, and discuss possible continuations.
\paragraph{Exponential localization.}
The convolution kernel $\K$ in our assumption is in general not exponentially localized. Therefore, the solution we obtained will not be exponentially localized in general. Consider the scalar equation $-u+K\ast u=\mu u -u^2$ to which our assumptions apply. If $u$ is an exponentially localized solution, then $u+\mu u-u^2 = K\ast u$ will also be exponentially localized, however, this is not possible since $K$ has merely algebraic decay.




\paragraph{Stability of the bifurcating branches.}
If we consider evolution equation of the form \[U_t = U+\K\ast U-\Nl(U;\mu), \hspace{0.1in}\text{ or }\hspace{0.1in} U_{tt} =U+\K\ast U-\Nl(U;\mu),\] then a natural continuation is to study the stability properties of the solutions bifurcated from the trivial state. The generic condition \eqref{muvCoe} and \eqref{QuadCoe} resembles the condition for a transcritical bifurcation in the equation $u^{''}=\mu u-u^2$, although we only assumed $\mu>0$ in this paper, a similar argument can be used to show nontrivial solutions bifurcate from $U=0$ as $\mu$ becomes negative, 

\appendix
\addcontentsline{toc}{section}{Appendices}
\section*{Appendices}
\section{Smoothness and estimates of the superposition operator}
For a smooth function $f \in \mathscr{C}^\infty(\R^k\times \R;\R^k)$ depending on a parameter $\mu$ and a Sobolev function $u  \in H^\ell(\R^n;\R^k)$, we define the superposition operator (or Nemytskii operator) $\tilde{f}$ as 
\[
\tilde{f}(u;\mu) (x) := f(u(x);\mu) \hspace{0.2in}(x \in \R^n).
\]

We prove the following results:
\begin{proposition}\label{spHest}
Suppose $\ell > n/2$ and $f(0;\mu)=0$, then $\tilde{f}$ takes $H^\ell(\R^n;\R^k)$ into itself and is bounded, $\|\tilde{f}(u)\|_{H^\ell} \le C=C(\rho)$, a constant depending on $\rho$ if $\|u\|_{H^\ell} \le \rho$. Moreover, $\tilde{f}$ is smooth as a map on $H^\ell(\R^n;\R^k)$.
\end{proposition}
\begin{Proof}We mostly follow \cite{runst1996sobolev}, the only differences is our function $f$ depend on a parameter $\mu$ and $f$ is defined on $\R^k$, which does not affect the proof.

Write $f=(f_1,\ldots,f_k)$, then without loss of generality we assume $f$ is scalar valued. We need to show $\tilde{f}(u;\mu)\in H^\ell$ for each $u \in H^\ell$, by definition, it is equivalent to show that $D^\alpha \tilde{f}(u;\mu) (x)$ are in $L^2$ for all $ |\alpha|\le \ell$.

First consider the case $\alpha = 0$, then we need to show the integral $\int_{\R^n} |f(u(x);\mu)|^2 dx
$ is finite. Indeed, by Sobolev embedding, $u(x) \in \mathscr{C}_b(\R^n;\R^k)$, since $f$ is smooth, by the mean value theorem, we have $|f(u(x);\mu)| \le C|u|$ for some constant $C$ and all $u$ with $|u|<\|u\|_{L^\infty}$, the desired estimate follows.

For the case $0<|\alpha|<\ell$, fix $i\in \{1,\ldots,k\}$ and $m\le |\alpha|$, we take the $m$th partial derivative of $f$ with respect to $x_i$, $\frac{\partial^m}{\partial x^m}\tilde{f}(u)(x)$. By the chain rule, it is a finite sum of multi-linear forms of the form
\[
D^\beta \tilde{f}(u;\mu) (x) \left[ \frac{\partial^{r_1}u}{\partial x_i^{r_1}},\ldots,\frac{\partial^{r_j}u}{\partial x_i^{r_j}} \right], 
\]
with $|\beta|=j \le m$ and nonnegative integers $r_1,\ldots,r_j$ satisfying $r_1+\cdots+r_j = m$. Again, by smoothness and Sobolev embedding, the form $D^\beta \tilde{f}(u;\mu)(x)$ is bounded for $x\in \R^n$, hence if we can show the following the proof will be complete
\[
\left\| \left|\frac{\partial^{r_1}u}{\partial x_i^{r_1}}\right| \cdots \left|\frac{\partial^{r_j}u}{\partial x_i^{r_j}}\right| \right\|_{L^2} \le C\|u\|_{H^\ell}^j.
\]
This estimate is done in section 5.2.4 of \cite{runst1996sobolev} for the case, relies heavily on the Holder inequality and Sobolev embedding. We conclude here that $\tilde{f}$ maps $H^\ell$ into itself and is bounded.

To show the smoothness of $\tilde{f}(u;\mu)$ in $u$, we suppress the dependence on $\mu$. By Taylor's theorem we can write
\[
f(u+v)(x) - f(u)(x)-Df(u(x))v  = 2\sum_{|\alpha|=2} \frac{v(x)^\alpha}{\alpha!} \int_0^1 (1-t) D^\alpha f(u(x)+tv(x))dt,
\]
apply the generalized Minkowski's inequality for integrals and the multiplication algebra property of $H^\ell$, we have
\begin{align*}
&\left\|\sum_{|\alpha|=2} v(x)^\alpha \int_0^1 (1-t) D^\alpha f(u(x)+tv(x))dt \right\|_{H^\ell}\\
 &\le \left\|\sum_{|\alpha|=2} v(x)^\alpha \int_0^1 (1-t)D^\alpha [f(u(x)+tv(x))-D^\alpha f(0)]dt\right\|_{H^\ell} + \left \|\frac{1}{2}\sum_{|\alpha|=2} D^\alpha(0) v(x)^\alpha \right\|_{H^\ell} \\
 &\le \| v(x) \|^2_{H^\ell}\left(\sup_{0\le t\le 1}\|\sum_{|\alpha|=2}D^\alpha f(u(x)+tv(x))-D^\alpha f(0)\|_{H^\ell} + c\max_{|\alpha|=2}|D^\alpha(0)|\right),
\end{align*}
for some constant $c$. Since $D^\alpha f(\cdot)-D^\alpha f(0):= g(\cdot)$ satisfies $g(0) = 0$ and is smooth, we apply the first part of this proof to conclude that $\|\sum_{|\alpha|=2}D^\alpha f(u(x)+tv(x))-D^\alpha f(0)\|_{H^\ell}$ is bounded by some constant $C=C(\rho)$ if $\|u\|_{H^\ell},\|v\|_{H^\ell} \le \rho$, which is independent of $t$. Hence $\tilde{f}$ will be Frechet differentiable provided that the map $u \mapsto Df(u(x))$ is continuous as a map from $H^\ell$ to $L(H^\ell)$, the space of bounded linear maps from $H^\ell$ to itself. To verify this, note first we indeed have $Df(u(x)) \in L(H^\ell)$ since $f$ is smooth and $u$ is bounded by Sobolev embedding. For continuity, fix $u,v,w \in H^\ell$, it suffices to show
\[
\|[Df(u)-Df(v)]w \|_{H^\ell} \le C\|u-v\|_{H^\ell}\|w\|_{H^\ell},
\]
holds for some constant $C$. However, since $f$ is smooth, $Df$ is locally Lipschitz, again using the multiplication algebra property and embedding, we readily conclude the above estimate holds.

The above arguments can be easily extended to show higher regularity of the map $\tilde{f}$, by application of higher order Taylor's formula, the Frechet derivative will be given by 
\[
\left( D^\alpha \tilde{f}(u) \right)[v_1,\ldots,v_n] (x)= D^\alpha f(u(x)) [v_1(x),\ldots,v_n(x)],
\]
which is a symmetric multilinear map from $\displaystyle \Pi_{i=1}^n H^\ell $ to $H^\ell$. The proof is finished.
\end{Proof}

Next we show estimates \eqref{Nlest} and \eqref{Liest} holds:
\begin{proposition}
For $\tilde{\Rm}_j$ defined in section 2.2, fix $u \in H^\ell(\R^n)$ and $v \in H^\ell(\R^n;\R^{k-1})$ with $\|u\|_{H^\ell}\le \rho, \|v\|_{H^\ell} \le \rho$ for some constant $\rho$, there then exist a constant $C=C(\rho)$ such that 
\[
\|\tilde{\Rm}_j(u,v;\eps)\|_{H^\ell} \le C(\rho) \eps^6, \hspace{0.3in}\|D_u \tilde{\Rm}_j(u,v;\eps)\|_{H^\ell \to H^\ell} \le C(\rho)\eps^6
\]
\end{proposition}
\begin{Proof}For this argument we may assume $\beta=-1$ and $\alpha=1$ to simplify the notations.

Recall that $\tilde{\Rm}_j(u,v,\eps) = \Rm_j(\frac{\eps^2 u}{-\beta}, \eps^2 v; \frac{\eps^2}{\alpha})$ and $\Rm_j = \Rm_j(p,q;\mu)=\Rm_j(U;\mu)$ satisfies \eqref{odR}, $|\Rm_j(U;\mu)| = \rmO(\mu^2|U|+\mu|U|^2+|U|^3)$ with $\mu =\eps^2$. 

Hence, we may write $\Rm_j(U;\mu) = T_j(U;\mu)r_j(U;\mu)$ where $T(U;\mu) = (\mu^2 a_jU+\mu b_j[U,U]+\mu [U,U,U])$ for some symmetric multilinear form (with vector values) $a_j$, $b_j$ and $c_j$, and $r_j$ is a smooth function. From the previous proposition, we know $r_j(U(\cdot);\mu)$ is smooth and bounded as a map on $H^\ell(\R^n)$ and $H^\ell(\R^n;\R^{k-1})$, using the multiplication algebra property, we have, for $V=(u,v)$
\[
\|\tilde{\Rm}_j(V)\|_{H^\ell} = \|\Rm_j(\eps^2V;\eps^2)\| \le \| \eps^6|V|+\eps^6|V|^2+\eps^6|V|^3 \|_{H^\ell} \|r_j(\eps^2V;\eps^2)\|_{H^\ell} \le C(\rho) \eps^6
\]
for some constant $C=C(\rho)$.

For the derivative term $D_u\tilde{\Rm}_j$, we know it is given by
\[
D_u\tilde{\Rm}_j(u,v;\eps) w (\cdot) = D_u \tilde{\Rm}_j(u(\cdot),v(\cdot);\eps) w(\cdot)
\]
for $w \in H^\ell$. Hence it is enough to show the estimate $\|D_u\tilde{\Rm}_j(u(\cdot),v(\cdot);\eps)\|_{H^\ell} \le C(\rho)\eps^6$. Write again $V=(u,v)^T$ and $U = \eps^2V$, then using Leibniz's rule we have
\[
D_V \tilde{\Rm_j}(V;\eps) = D_U \Rm_j(\eps^2V;\eps^2) = \eps^2\left[D_U(T(U;\mu))r_j(U;\mu)+ (T(U;\mu))D_U r_j(U;\mu)\right],
\]
note that $D_U(T(U;\mu))$ satisfy $|D_UT(U;\mu)| \le C(\mu^2+\mu|U|+|U|^2)$ for some constant $C$, since $r_j$ is smooth, we conclude again by the multiplication algebra property that
\[
\| D_V\tilde{\Rm}_j(V(\cdot);\eps)\|_{H^\ell} \le C(\rho)\eps^6.
\]
This in particular implies $\|D_u\tilde{\Rm}_j(u,v;\eps)\|_{H^\ell \to H^\ell} = \rmO(\eps^6)$, the proof is finished.
\end{Proof}

\begin{lemma} \label{Lest}
For the linear operator $L^\eps$ defined in section 2.2, we have
\[
\| L^\eps - S^0\|_{H^\ell\to H^\ell} \to 0, \hspace{0.3in}\text{ as }\eps \to 0.
\]
\end{lemma}
\begin{Proof}
Now, if $u \in H^\ell$ and $\delta>0$ is arbitrary, using the Fourier transform characterization of $H^\ell$, we compute
\begin{align*}
\|(L^\eps - S^0)u\|_{H^{\ell}} &= \|(S(\eps s)-S(0))\widehat{u}(\xi)(1+|\xi|^2)^{\ell/2}\|_{L^2} \\
& = \int_{|\xi|<R} (S(\eps |\xi|)-S(0))\widehat{u}(\xi)(1+|\xi|^2)^{\ell/2}d\xi + \int_{|\xi| \ge R} (\cdots)\\
& := I_1 + I_2,
\end{align*}
where $R$ is chosen so that $I_2 < \delta$, which is possible since $S$ is bounded and $v \in H^\ell$. For $I_1$, since we are integrating $|\xi|\le R$, we may choose $\eps$ small so that $|S(\eps|\xi|)-S(0)|<\delta/\|u\|_{H^\ell}$ since $S$ is continuous at $0$. Hence $\|(L^\eps-S^0)u\|_{H^\ell} < \delta$, which is the desired conclusion.
\end{Proof}

\section{Properties of the ground state solution}
Here we collect the relevant information about the ground state solution to the semilinear elliptic equation $-\Delta u + u - u^p = 0$. Most of the results can be found in the Appendix of \cite{wei2013mathematical}).

\begin{lemma}\label{GS}
Let $2<p<\infty$ if $n=1,2$ and $1<p<\frac{n+2}{n-2}$ if $n>2$. Then there exist a positive, radially symmetric function and decreasing function $w=w(r)=w(|x|)$ which solves
\begin{equation}\label{gs}
-\Delta w +w-w^p =0, \hspace{0.2in}\text{ in }\R^n, 
\end{equation}
with the asymptotic estimates
\begin{equation}\label{GsAsy}
\lim_{r\to \infty} u(r)r^{\frac{n-1}{2}}e^r = C,
\end{equation}
where $0<C<\infty$ is a constant.
 Moreover, any nonnegative weak solution $0\neq u \in H^1(\R^n)$ of \eqref{GS} is given by $u(x) = w(x-a)$ for some $a \in \R^n$, and the linearization at $w$ is nondegenerate in the following sense: the operator $\cL = -\Delta+1 - pw^{p-1} $, considered as a self-adjoint operator on $L^2(\R^n)$ satisfies
 \[
 \ker \cL =\spa \{ \partial_{x_1}w,\ldots,\partial_{x_n}w \}
 \]
\end{lemma}
We also need the following result about the spectrum of $\cL$.
\begin{lemma}\label{spec}
The eigenvalue problem $\cL\phi = \lambda \phi$ with $\phi \in H^1(\R^n)$ satisfies 
\[
\lambda_1<0, \hspace{0.2in}\lambda_2 =\cdots=\lambda_{n+1} =0, \hspace{0.2in} \lambda_{n+2}>0.
\]
\end{lemma}
Using the fact that $w$ is radial, we can convert \eqref{gs} to the ODE
\[
-\partial_r^2 w - \frac{n-1}{r} \partial_r w +w - w^p = 0,
\]
using \ref{GsAsy} and the standard regularity and decay argument, we have $w \in \mathscr{C}^\infty(\R^n)$ and decays exponentially together with all of its derivatives. In particular, we have
\begin{lemma}
 The unique ground state solution $w$ of \eqref{GS} belongs to $ H^\ell(\R^n)$ for $\ell=0,1,2,\ldots$.
\end{lemma}

Using these facts, consider now the equation
\begin{equation}
-d\Delta v + v- v^2 = 0 \hspace{0.2in}\text{ in }x \in \R^n, \label{MdEq}
\end{equation}
with $d>0$. If $v$ solves \eqref{MdEq}, Set $v^{\sqrt{d}}(x) = v(\sqrt{d} x)$, then $v^{\sqrt{d}}$ solves \eqref{GS} with $p=2$, provided $n<6$. Hence $v^{\sqrt{d}}$ is a translation of $w$, therefore there exist a unique solution to \eqref{MdEq} which we denote by $v_*$.

Hence, any properties of the linearized operator $\cL^d = -d\Delta +1 -2v_*$ will be the same with that of $\cL$. In particular, we have the following 




\begin{lemma}\label{Nondegen}
The operator $\cJ$ is bounded invertible from  $H^{\ell}_r(\R^n)$ to $H^{\ell-2}_r(\R^n)$ for $\ell>n/2+1$.
\end{lemma}
\begin{Proof}
From the nondegeneracy of $\cL$, we know the kernel is spanned by the $n$ partial derivatives of $v_*$. We claim that $(\ker \cL )\cap L^2_r(\R^n) = \{0\}$. The same will hold for $\cJ$ from our rescaling arguments.

To see this, we follow the strategy of \cite{chang2007spectra}, first, by \ref{spec} and the Courant nodal domain theorem, if $\phi \in L^2_r(\R^n)$ is an eigenfunction of $\cL$ correspond to the second eigenvalue $0$, then it changes sign exactly once, without loss of generality, we pick $r_*$ so that $\phi(r)>0$ for $r\in (0,r_*)$ and $\phi(r)<0$ for $r \in (r_*,\infty)$.

Then, as in [Spectra of Linearized Operators for NLS Solitary Waves] we find 
\[
\cL w = (1-p)w^p, \hspace{0.4in} \cL \eta = -2 w
\]
where $\eta = \frac{2}{p-1}w+x\cdot \nabla w$. Since $\cL$ is self adjoint on $L^2(\R^n)$, we have $\int \phi w^p=\int \phi w =0$. But then $\int \phi(w^p-w(r_*)^{p-1}w )=0$, while the integrand is positive by our choice of $r_*$ and the monotonicity of $w$. This contradiction shows $\ker \cL$ is trivial when restricted to $L_r^2(\R^n)$.

Next we show that $\cL$ is Fredholm with index zero from $H^\ell(\R^n)$ to $H^{\ell-2}(\R^n)$ by writing it as the compact perturbation of an invertible operator. For $h \in H^\ell(\R^n)$, we decompose $\mathcal{L}$ as
\[
\cL_1h = -\Delta h + h, \hspace{0.2in} \cL_2 h = (2v_*)h.
\] 

We first show $\cL_2: H^\ell(\R^n) \to H^\ell(\R^n)$ is compact. Fix an integer $k>0$, let $\chi_k$ be a positive smooth cutoff function which equals $1$ on the cube $[-k,k]^n$ and vanishes outside of $[-2k,2k]^n$. Consider the sequence of operators $\cL_2^{(k)} : H^\ell(\R^n) \to H^{\ell-2}(\R^n)$ defined by $\cL_2^{(k)} h = \chi_k (2v_*)h$. We claim that $\cL_2^{(k)}$ are compact operators for each $k$ and converges to $\cL_2$ in the operator norm.

To see $\cL_2^{(k)}$ are compact for each $k$, let $\{h_i\}_{i=1}^{\infty}$ be a bounded sequence of functions in $H^{\ell}(\R^n)$ with $\|h_i\|_{H^2} 
\le 1$. We want to show that there exist a subsequence $h_{i'}$ so that 
$\cL_2^{(k)} h_{i'} = \chi_k (2v_*)h_{i'}$ is convergent in $H^{\ell-2}(\R^n)$. From the Sobolev embedding $H^{\ell}(\R^n) \to 
\mathscr{C}^1_0(\R^n)$, the $h_i$ and their derivatives are uniformly 
bounded on $\R^n$. Therefore, $\cL_2^{(k)}h_i = \chi_k(2v_*)h_i$ is a 
sequence of functions that is uniformly bounded and equicontinuous on 
the compact set $[-2k,2k]^n$, we conclude from the Arzela-Ascoli 
theorem that there exist a subsequence $i'$ such that $\cL_2^{(k)}h_{i'}$ 
converges uniformly on $[-2k,2k]^n$. Since $\cL_2^{(k)}h_{i'}$ are 
continuous and have a common compact support $[-2k,2k]$, it is an 
convergent sequence in $H^{\ell-2}(\R^n)$ as well. Therefore $\cL_2^{(k)}$ is compact for each $k$.

To see $\cL_2^{(k)} \to \cL_2$ in the operator norm, fix $h \in H^\ell(\R^n;\R)$ with $\|h\|_{H^\ell} =1$, we have
\[
\|(\cL_2^{(k)}-\cL_2)h\|_{L^2} = \|(\chi_k-1)(2v_*)h\|_{L^2}\le  \sup_{x\in \R^n}(\chi_k(x)-1)(2v_*(x)) \|h\|_{L^2}.
\]

Since $\chi_k = 1$ on $[-k,k]^n$ and $v_*(x) \to 0$ as $|x|\to \infty$, it follows that $\displaystyle \sup_{x\in \R^n}(\chi_k(x)-1)(2v_*(x)) \to 0$ as $k \to \infty$. Thus $\|\cL_2^{(k)} - \cL_2\|_{H^\ell \to H^{\ell-2}} \to 0$ and we conclude that $\cL_2:H^\ell(\R^n) \to H^{\ell-2}(\R^n)$ is compact.

The fact that $\cL_1  = -\Delta  +1 $ is invertible from $H^\ell (\R^n)$ to $H^{\ell-2}(\R^n)$ follows directly by looking at its Fourier symbol. This shows $\cL$ is the compact perturbation of an invertible operator, by standard Fredholm theory [Kato.....or other Functional analysis book], it follows that $\cL$ is Fredholm with index $0$ from $H^\ell(\R^n)$ to $H^{\ell-2}(\R^n)$.

Therefore, when restricted to the subspace $H^\ell_r(\R^n)$, which has trivial intersection with $\ker \cL$ as shown earlier, $\cL$ becomes invertible as an operator from $H^\ell_r(\R^n)$ to $H^{\ell-2}_r(\R^n)$, the same must also hold for $\mathcal{J}$, which concludes the proof.
\end{Proof}
\bibliographystyle{plain}
\bibliography{nlBfCs}


\end{document}
